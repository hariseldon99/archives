\documentclass[a4paper,9pt]{article}

\usepackage[defaultsans]{droidsans}
\renewcommand*\familydefault{\sfdefault} %% Only if the base font of the document is to be typewriter style
\usepackage[latin1]{inputenc}
\usepackage{fancyhdr}
\usepackage{eurosym}
\usepackage{lastpage}
\usepackage[margin=0.6in]{geometry}
\usepackage[square, comma, numbers, sort&compress]{natbib}
\usepackage{hyperref}

\def\Acronimo{CCQS}
%CCQS == Coarsening in Classical and Quantum Systems
\hypersetup{
    pdftitle={\Acronimo{}},    % title
    pdfauthor={Analabha Roy},
    colorlinks=true,
    citecolor=black,
    linkcolor=black,
    urlcolor=blue
  }


\pagestyle{fancy}
\fancyhead[l]{\Acronimo{} - Analabha Roy \\Santa Fe Institute 
}
\fancyhead[r]{Omidyar Fellowship Program - 2013}


\renewcommand{\headrulewidth}{0pt}
%\renewcommand{\thesection}{\thesection}
\def\thesection{B\arabic{section}} 


\let\oldthebibliography=\thebibliography
  \let\endoldthebibliography=\endthebibliography
  \renewenvironment{thebibliography}[1]{%
    \begin{oldthebibliography}{#1}%
      \setlength{\parskip}{0ex}%
      \setlength{\itemsep}{0ex}%
  }%
  {%
    \end{oldthebibliography}%
  }



\title{Coarsening in Classical and Quantum Systems: Research Proposal for the SFI-Omidyar Fellowship}
\author{Analabha Roy}


\begin{document}
\maketitle
 
\section{Introduction:}
\label{sec:sciTecQuality}

The main objective of my proposed research work is to understand the dynamics of \textbf{Coarsening} in \textbf{Classical} and \textbf{Quantum} many body \textbf{Systems}
(referred to in this proposal by the acronym '\textbf{CCQS}') while they are approaching equilibrium after a quench in a  {parameter}; a particularly important aspect of the nonequilibrium dynamics seen in recent experiments. The focus lies on classical nonequilibrium dynamical systems in a closed field theory with a double well potential, the classical evolution of an open system like coupled micromagnets, and the nonequilibrium dynamics of quantum-quenched dissipative spin heterostructures. A 'quench' is defined as the \textit{diabatic} variation of a thermodynamic or other  {parameter} of the system, such as temperature, mass or chemical potential. In a diabatic variation, the rate at which the quench rate is faster than all the relaxation rates intrinsic to the system, thus allowing for approximating the change as instantaneous. 
Coarsening after a quench constitutes the dynamical process by which the characteristic  {size} of the support of the equilibrium phases grows.

\section{Methodology and Approach:}

The first part of the proposed project involves the analytical and numerical study of classical quenches in $\phi^4$ theory in  {$d$} 
dimension. The potential in the free energy will be constructed to $4^{th}$ order in the order parameter $\phi$. The evolution of the free-energy landscape with the control parameter driving a phase transition guides the understanding of the post-quench dynamics from, typically, a disordered phase to an ordered phase. If the free energy lacks a cubic term, then phase transitions are of second order, driven by instability at criticality. If the free energy has cubic terms, then phase transitions are of first order, driven by metastability at criticality. The project involves the study of quenches to criticality, as well as sub-critical quenches by investigating the microscopic dynamics of the order parameter. This dynamics is governed by a quasi-Newtonian equation of motion. Other descriptions, such as the field 
large-$\mathcal{N}$ approximation, will also be considered. After an instantaneous quench at $t=0$ the subsequent evolution of the order parameter is performed by integrating the equation of motion with the post-quenched parameters  {(the double-well structure)}. The temporal behavior of the equilibrating field will be studied. During this \textit{coarsening} process, space-time correlations allow for the identification of a growing length scale. Domains of equilibrium states are expected to grow with this length scale, and a spatial profile of 'kinks' or  {`domain walls'} that demarcate these regions is expected to provide insights into the coarsening process. Domain walls will be identified, whose gradients are expected to drive coarsening. Coarsening is thus expected to drive the nucleation and growth of domains that support the equilibrium phase. Near criticality, divergences of time scales via critical slowing down is also known to occur, and the scaling behavior of critical quenches will also be studied analytically using scaling and renormalization group arguments. In addition, the system is expected to build fractal clusters~\cite{fractal}
and the equilibrium and nonequilibrium contributions are multiplicatively separated. In sub-critical quenches, the asymptotic behavior of the characteristic coarsening scale relative to equilibrium correlations is expected to be governed by the dynamic scaling hypothesis~\cite{dynscal}, where the domain structure is statistically independent of time when lengths are scaled accordingly. The goals of this phase of the project are the profiling of the universal scaling laws described above, as well as the evolution of structure interfaces in the order parameter field through the kinks in the solution.

The next phase of the project will involve studying the dynamics of open classical systems \textit{viz.} systems connected to a thermal reservoir. Classical $\phi^4$ theories of the type discussed above can be linked to closed Ising magnets, and open systems of magnets can be studied by dealing with the interactions between magnetic moments on sub-micrometre length scales. These are governed by competition between the dipolar energy and the exchange energy, the outcome of which governs the long range magnetic order, if any. In such systems, the Landau-Lifshitz-Gilbert (LLG) equation is a key model for describing the nonlinear evolution. The micromagnets in this model are built up from fermions acting under a time-dependent Zeeman field, and the Ehrenfest dynamics therein~\cite{gll:review}, together with a phenomenological damping term that takes into account the saturation of the magnetization. The dynamics of many coupled LLG systems connected to a thermal bath (canonical ensemble at equilibrium) can be formulated from here. In the continuum limit, the result is a nonlinear partial differential equation in the spin field. The ensuing dynamics links the physics to the microscopic Hamiltonian structures of spin lattices, including planar XX and XY structures~\cite{laxmanan:xxxy} whose quantum evolution will be studied later. This phase of the project aims to formulate the LLG dynamics after a quench across the magnetic phase transitions, and investigate the transition to equilibration in a manner similar to that described in the paragraph above. The dynamics in these systems have wide interdisciplinary relevance. They have intimate connections with many of the well-known integrable soliton equations, including nonlinear Schr\"odinger and sine-Gordon equations~\cite{laxmanan:xxxy,sinegordon}. The possibility of classical chaos in such systems~\cite{gll:review} leads it to other disciplines, such as power generation and synchronization~\cite{lax13} and inhomogeneous filaments~\cite{lax14}. 

The logical continuation of this project now draws attention to the quantum realm in such systems. At zero temperature, a closed quantum system is said to be in equilibrium when it is at the ground state (or generally speaking, any eigenstate) of the Hamiltonian of the system. At finite temperatures, a system at equilibrium can no longer be found in a single eigenstate, but is delocalized in the Hilbert space over many eigenstates with a thermal probability distribution. A \textit{diabatic quench} causes a  {parameter} to vary rapidly in comparison to the relaxation times of the excitations, both thermal and quantum. The nonequilibrium dynamics of such closed quantum systems after a quench, although an ongoing subject of study,  {are somewhat better understood than their open counterparts}, and involve well-understood principles~\footnote{principles such as dephasing~\cite{thermalization}, Eigenstate Thermalization hypothesis~\cite{thermalization,krishrev}, Kibble-Zurek mechanism~\cite{bikashbabu}, Landau-Zener tunneling~\cite{bikashbabu}, semiclassical Nonlinear Schr\"odinger equations~\cite{colrev,rammer}, Keldysh theory~\cite{gorkov, volkov}, and others}. The quantum dynamics part of this research project will involve the application of some of the above-mentioned ideas, as well as newer stochastic methods, to \textit{open quantum systems} \textit{viz.} systems that can exchange energy via connections to a reservoir of heat. Away from $T=0$, the pure quantum dynamics has to be weighed by the thermal probabilities of the initial state that is presumed to be in thermal equilibrium. For open systems, such dynamics can be studied by completely specifying the nature of the reservoir, and treat the coupled system + reservoir as a closed 
quantum system, or by treating the reservoir stochastically. The nonequilibrium dynamics, thus formulated, allows the profiling of local quantum correlations and coarse-grained responses whose spatial extent described coarsening that is expected to grow towards the steady state. Analytics will involve modeling such systems by \textit{spin chain heterostructures}~\cite{arrachea},  constructed by linking finite or semi-infinite XX or XY spin chains (of various anisotropies) to each other at their ends. The dissipative dynamics of the central chain after a quench can thus be obtained from the Hamiltonian quantum dynamics of the entire system of links. Such systems can be mapped onto p-wave BCS superconducting fermions, and the nonequilibrium Dyson equation can be formulated in a manner similar to~\cite{gorkov, volkov}. Post-quench dynamics can be studied by solving the resultant equations for the local fermion correlations by diagrammatic approximations to the nonequilibrium self-energy, either in the collisionless regime~\cite{volkov,ncnsd2012}, or in the regime where collisions are rapid enough to have relaxed away and only order parameter dynamics remains~\cite{ncnsd2012}. The scaling laws for the responses in such dynamics will also be obtained.  In the fast collision regime, the dynamics can be approximated to be near equilibrium, with a classical $\phi^4$ approximation in the path integral (see~\cite{colrev} and the references therein). This allows for a connection with the classical problem discussed above, and similar numerical methods can be used. Evaluating the Goldstone modes by a linear 
stability analysis of the dynamics around equilibrium will also be done so as to profile excitations and fluctuations about the mean field.

The approach detailed above, with the full Hamiltonian description of the system and reservoir, comes at the price of numerous analytical and computational difficulties. The final part of this project will deal with alternative formalisms of open quantum systems, where the reduced density matrix is evolved in time by a master equation, allowing for the inclusion of incoherent processes which represent interactions with a reservoir. Such dynamics is regularly studied in quantum optics using the \textbf{Kossakowski-Lindblad equation}, where it can represent absorption or emission from a reservoir~\cite{lindblad}. Here, the regular Liouvillean dynamics of closed quantum density matrices are dressed with Lindblad bath operators which act locally on degrees of freedom near each bath. Although this approach is not universally valid, it is a reasonable starting point to study Heisenberg chains such as the ones being studied in this project~\cite{spinchains:lindblad}. Here, as in the previous paragraph, the ends of 
a finite spin chain are coupled to canonical Lindblad spin operators whose amplitudes are determined by the thermodynamics of the reservoir.  This approach is computationally simpler, and the density matrix for such a mixed system can be solved using DMRG methods. The dynamics of coarsening in Lindblad systems can be solved after a quench in this manner.

 
\section{Relevance and interdisciplinary nature of the project}
\label{sec:timeliness}

Coarsening is a very basic aspect of non-equilibrium quantum dynamics. Understanding it would shine light on a vast landscape of quantum phenomena ranging from the process of defect generation in critical quantum relaxation~\cite{relaxation}, thermalization of a closed many-body quantum system~\cite{krishrev, thermalization}, thermalization and effective temperatures of open quantum systems~\cite{thermopen}, glassy systems~\cite{glassy}, mesoscopic systems~\cite{meso}, to the operation of near future quantum devices like an analog quantum computer~\cite{annealing}. The potential and the target of our project, as well as its methodologies, are thus genuinely interdisciplinary and of very broad interest. In general, nonequilibrium dynamics of many body systems provide deep insights into several complex phenomena in nature, ranging from the behavior around phase transitions in bulk matter, biological systems,  to the creation of the known forces of the universe. In addition, studies in Quantum Annealing indicates that there is a very deep relationship between different aspects of quantum non-equilibrium dynamics and the basic limitations of a quantum computer~\cite{annealing}. Nonequilibrium dynamics of open quantum systems are an important part of quantum optics, quantum measurement theory, quantum statistical mechanics, quantum information science, quantum cosmology and semiclassical approximations~\cite{openq}. 

Due to the many implications and interdisciplinary nature of nonequilibrium dynamics, a comprehensive study of the approach to equilibrium in such systems would provide insights into the behavior of several nonequilibrium systems that are actively studied in academia. Also, the LLG equations are related to the dynamics of several important physical systems (see~\cite{gll:review} and references therein). The classical kinetics of systems undergoing critical dynamics or an ordering process is an important problem for condensed matter physicists, and enhances the generic understanding of phenomena not fully understood, such as pattern formation in nonequilibrium systems and the approach to equilibrium in systems with slow dynamics. In addition, most treatments of open quantum systems out of equilibrium have involved phenomenological approaches using generic structures~\cite{openspin}, restricting  themselves to the equilibrium steady state, and often performed within the linear response regime with small temperature gradients. Therefore, it is desirable to develop alternative approaches designed to treat systems far out of equilibrium while they are coarsening towards steady states. Finally, the detailed study of quantum nonequilibrium dynamics has, for the most part, been restricted to closed quantum systems, and studies of open systems have begun fairly recently~\cite{daley}. Thus, studies of coarsening in open quantum systems will contribute towards a fledgling area of research. 
\begin{thebibliography}{}

\bibitem{fractal}
J.D. Gunton, M. San Miguel, and P.S. Sahni, in \textit{Phase Transitions and Critical Phenomena}, C. Domb
and J.L. Lebowitz eds. (Academic Press, New York, 1983)  {\bf 8} 267; H. Furukawa, Adv. Phys. {\bf 6}, 703 (1985);
J. Langer, in \textit{Solids Far From Equilibrium}, C. Godr'eche ed. (Cambridge University Press, Cambridge, 1992).

\bibitem{dynscal}
B.I. Halperin and P.C. Hohenberg, Phys. Rev. {\bf 177}:2, 952 (1969).

\bibitem{gll:review}
M. Lakshmanan,  Phil. Trans. R. Soc. A {\bf 369}:1939 1280-1300 (2011).

\bibitem{laxmanan:xxxy}
M. Lakshmanan and A. Saxena, Physica D {\bf 237} 885-897 (2008); J. A. G. Roberts and C.J. Thompson, J. Phys. A {\bf 21} 1769-1780 (1988).

\bibitem{sinegordon}
M. Daniel and L. Kavitha, Phys. Rev. B {\bf 66} 184433 (2002); H.J. Mikeska and M. Steiner, Adv. Phys. {\bf40}:3 191-356 (1991). 

\bibitem{lax13}
J. Grollier, V. Cross  and A. Fert, Phys. Rev. B {\bf 73} 060409 (2006).

\bibitem{lax14}
Y.B. Bazaliy, B.A. Jones and S.C. Zhang, Phys. Rev. B {\bf 69} 094421 (2004).

\bibitem{thermalization}
M. Rigol, V. Dunjko, and M. Olshanii, Nature 452, 854 (2008).

\bibitem{krishrev}
A. Polkovnikov, K. Sengupta, A. Silva, M. Vengalattore, Rev. Mod. Phys. \textbf{83}, 863 (2011).

\bibitem{bikashbabu}
A. Dutta, U. Divakaran, D. Sen, B.K. Chakrabarti, T.F. Rosenbaum, G. Aeppli, arXiv:1012:0653 (unpublished).

\bibitem{colrev}
A. Roy, Eur. Phys. J. {Plus}, {\bf 127}:3, 34 (2012).

\bibitem{rammer}
J. Rammer, \textit{Quantum Field Theory of Non-equilibrium States} (Cambridge University Press, Cambridge 2007).

\bibitem{gorkov}
L.P. Gorkov, G.M. Eliashberg, Sov. Phys. JETP \textbf{27}, 328 (1968).

\bibitem{volkov}
A.F. Volkov, Sh.M. Kogan, Sov. Phys. JETP \textbf{38}(5), 1018 (1974).

\bibitem{arrachea}
L. Arrachea, G. S. Lozano, and A. A. Aligia, Phys. Rev. B {\bf 80}, 014425 (2009).

%\bibitem{imry}
%Y. Imry, \textit{Introduction to Mesoscopic Physics}, (Oxford University Press, 1997).

\bibitem{ncnsd2012}
A. Roy, Invited mini-review (NCNSD $2012$),
Eur. Phys. J. ST, {\bf 222} (3-4), 975-993 (2013).

\bibitem{lindblad}
A. Kossakowski, Rep. Math. Phys. {\bf 3} 247 (1972); G. Lindblad , Commun. Math. Phys. {\bf 48} 119 (1976).

\bibitem{spinchains:lindblad}
S. Clark, J. Prior, M. J. Hartmann, D. Jaksch, and M. B. Plenio, New J. Phys. {\bf 12}, 025005 (2010).

\bibitem{relaxation}
W. H. Zurek, U. Dorner, and P. Zoller, Phys. Rev. Lett. {\bf 95}, 105701 (2005).

\bibitem{thermopen}
A. Caso, L. Arrachea, G. S. Lozano, Eur. Phys, J B, {\bf 85}:266, (2012).

\bibitem{glassy}
L. F. Cugliandolo and J. Kurchan, Phys. Rev. Lett. {\bf 71}, 173-176 (1993) 

\bibitem{meso}
L. Arrachea and L. F. Cugliandolo, Europhys. Lett. 70 642 (2005).

\bibitem{annealing}
Arnab Das and B. K. Chakrabarti Eds., \textit{Quantum Annealing and Related Optimization Methods}, Lecture Note in Physics, {\bf 679}, Springer-Verlag, Heidelberg (2005).

\bibitem{openq}
Breuer, Heinz-Peter; F. Petruccione. \textit{The Theory of Open Quantum Systems}, (Oxford University Press 2007).

\bibitem{openspin}
M. Michel, O. Hess,  H. Wichterich and J. Gemmer, Phys. Rev. B {\bf 77}, 104303 (2008) 

\bibitem{daley}
W. Yi, S. Diehl, A. J. Daley and P. Zoller, New J. Phys. {\bf 14}, 055002 (2012).

\end{thebibliography}

\end{document}
