\chapter{Conclusion}
\index{Conclusion@\emph{Conclusion}}%
\label{chapter-conclusion}
We have analyzed the dynamics of interacting two-boson systems for ultracold alkali metal atoms in electromagnetic traps. We have modeled double well potentials for those traps using magnetic confinement in atom chips as well as far-off resonance traps in optical lattices. The reduced atom Hamiltonian, when subjected to time-modulated sinusoidal drives, will undergo excitations to higher states, the transition to which can be coherently controlled by stimulated Raman adiabatic passage (STIRAP).

We have observed some unusual behavior in the classical and quantum dynamics of two bosons in a double well, where the interaction between them is weakly repulsive (the 'single particle regime'). Chaos in the separatrix region of the classical version of the coupled system corresponds with regions of high probability in the quantum Poincare map. However, a noticeable tunneling has been observed from the separatrix into the individual wells. We have also demonstrated the feasibility of a controlled excitation of the system into a higher energy state using STIRAP using external radiation pulses that cause dipole excitations. Avoided crossings in the Floquet eigenphases arise due to level repulsion caused by a loss of symmetry as they evolve adiabatically with time, and are connected with the dynamics of the underlying classical system. The classical dynamics of the system in those regions of the parameter space are nonintegrable, and the presence of multiple nonlinear resonances cause transitions to chaos from KAM tori~\cite{reichl}. A system undergoing a chaos-assisted adiabatic passage by avoiding these crossings will manifest quantum effects due to the classical chaos~\cite{reichl}~\cite{latka:avoidedcrossings:chaos}. Thus, the driving fields induce nonlinear resonances and chaos in localized regions of the phase space that affect the structure of the Floquet eigensystem and influence the STIRAP dynamics accordingly~\cite{na-reichl:pbox}. We have looked at the controlled dynamics of STIRAP from the ground state of such a system to the third excited state for small and large STIRAP pulse amplitudes. 

For sufficiently large amplitudes, the effects of the underlying classical dynamics start to manifest themselves through small avoided crossings between the involved Floquet eigenphases.  The first avoided crossing causes a change in character between the ground state and a linear combination of the other two states states that are connected by the STIRAP pulses. However, the second avoided crossings (of the same type as the first one) between the same eigenphases causes the character change to reverse. Thus, these avoided crossings cause a temporary loss in coherence of the wavefunction as it evolves from the ground state, after which the system proceeds to complete the dynamics as expected for an ordinary three level STIRAP, completely populating the final state of the system. The time scale (of the STIRAP pulses) for such avoided crossings to be avoided (so that the chaos assisted adiabatic passage that causes the character changes may occur) was computed from the numerical data using the Landau-Zener formula. 

The required pulse times for such a system turned out to be very slow at $1.62 \times 10^5$ units of $T_u$, where $T_u=\frac{2mL^2_u} {\hbar}$ for atoms of mass $m$ (chosen to be that of $^{85}Rb$ alkali metal atoms) and a quartic double well system with well minima at length $L_u$. This was confirmed by numerical simulations of the exact quantum dynamics, where the population transfers were seen for small pulse times and the effects of the avoided crossings seen for larger pulse times. If we use double wells of $L_u\simeq 50$ $nm$, then $T_u=6.68118$ $\mu s$, making the time scale for the avoided crossings to be $1.08235$ seconds. For optical systems, the double wells would be at least one order of magnitude larger, making the time scale two orders larger (nearly $2$ minutes). We also looked at STIRAP in a slightly adjusted well depth for which the chaos produced by additional resonances produce avoided crossings that can cause coherent population transfer to a higher state (in our case, the sixth excited state). This defect was verified for the exact time evolution of the system as well. The physical time scale for this crossing to be avoided was determined to be $1.2 \times 10^3$ units of $T_u$, which, for  $L_u\simeq 50$ $nm$, translates to $8.01742$ $ms$. For optical lattice systems, these time scales would increase by two orders of magnitude.  Thus, we demonstrated radiation pulses can be used to exert coherent control of the coupled boson system through chaos assisted adiabatic passages, just as has been recorded for systems with lower degrees of freedom.

We then proceeded to look at a similar dynamical system in an optical lattice generated by two counter-propagating lasers, except with stronger repulsive interaction. Time modulations in the frequencies of these lasers produce non-dipole excitations that can be controlled in a manner similar to our earlier system. The STIRAP pulses were tuned to connect very high energy states (the final state being the fourteenth excited state). The presence of an additional resonance with the seventeenth excited state, along with avoided crossings between the other states connected by STIRAP, cause the outcome of STIRAP to differ from the traditional three level case in a way similar to our previous system. We noticed two avoided crossings, one between the ground state and the fourteenth excited state that causes them to switch character with a time scale of $7.334 \times 10^4$ units of time (The characteristic time scale here is $1/(4\omega_r)$, where $\omega_r$ is the recoil frequency of the optical lattice and can be calculated from the wavelength, which is slightly detuned away from the $D_2$ transition line of $^{85}Rb$ as $1.03 \times 10^{-5}$ seconds). Thus, the physical time scale for this crossing is $0.756$ seconds. After this crossing is avoided, another crossing was seen between the ground state (now with character switched) and one of the states connected by STIRAP (a linear combination of the third and fourteenth excited states). The time scale for this crossing to be avoided was calculated for the $D_2$ transition line to be $0.225$ seconds. 

Thus, both crossings are avoided if the first one is avoided, resulting in a complete loss of coherence in the final outcome. STIRAP pulses with faster time scales will cause these crossings to be crossed with no change in eigenstate character. Thus, traditional three-level stirap will be seen, resulting in a  coherent population transfer to the fourteenth excited state. This effect of the underlying classical chaos will prove useful in understanding the coherent dynamics of excitations in systems of optical lattices that have been replicated in the laboratory, and is vital to the outcome of experiments involving coherent acceleration of bosons in optical lattices.