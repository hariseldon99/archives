\documentclass[a4paper,11pt,color]{article}

%% A template for the IEF Marie Curie action
%
% All very simple code, and standard packages. The bibliography uses
% IEEEtranSA style, which is very similar to alpha.
%
% The ethical issues tables in section B6 could be forced in place
% more elegantly, but it worked for me.
%
% DOUBLE CHECK the details of your call before using this
% template. The name of the sections and subsections changes from call
% to call, and new sections are added and removed.
%
% August 2010, v1.0 - Jesus Nuevo-Chiquero.
%
% This file is provided AS IS, with absolutely no warranty of
% anything. You are welcome to use it, but you assume all risks.
\usepackage[defaultsans]{droidsans}
\renewcommand*\familydefault{\sfdefault} %% Only if the base font of the document is to be typewriter style
\usepackage[latin1]{inputenc}
\usepackage{fancyhdr}
\usepackage{eurosym}
\usepackage{lastpage}
\usepackage{tabularx}
%\usepackage{xspace}
%\usepackage{lineno}
%\linenumbers
%\usepackage{draftwatermark}
%\SetWatermarkScale{7}

\usepackage{setspace}

\usepackage{multicol}
\usepackage{multirow}
\usepackage{color}
\usepackage{colortbl}
\usepackage{xcolor}
\usepackage[square, comma, numbers, sort&compress]{natbib}

\usepackage{hyperref}
\voffset=-2.5cm
%\hoffset=-0.5cm
\textheight=24cm
%\textwidth=17cm
\headheight=14pt
\textwidth=16cm
\oddsidemargin=0pt
\evensidemargin=0pt

\def\Acronimo{CCQS}
%CCQS == Coarsening in Classical and Quantum Systems
\hypersetup{
    pdftitle={\Acronimo{}, FP7-PEOPLE-IIF-2013},    % title
    pdfauthor={Analabha Roy},
    colorlinks=true,
    citecolor=black,
    linkcolor=black,
    urlcolor=blue
  }


\pagestyle{fancy}
\fancyfoot[C]{Part B - Page \thepage~of \pageref{LastPage}}
\fancyhead[l]{\Acronimo{} - Analabha Roy \\Marie Curie Actions
}
\fancyhead[r]{University Pierre et Marie Curie \\International Incoming Fellowship FP7-PEOPLE-IIF-2013}


\renewcommand{\headrulewidth}{0pt}
%\renewcommand{\thesection}{\thesection}
\def\thesection{B\arabic{section}} 


\let\oldthebibliography=\thebibliography
  \let\endoldthebibliography=\endthebibliography
  \renewenvironment{thebibliography}[1]{%
    \begin{oldthebibliography}{#1}%
      \setlength{\parskip}{0ex}%
      \setlength{\itemsep}{0ex}%
  }%
  {%
    \end{oldthebibliography}%
  }



% \title{Part B}
% \author{}
% \date{}

\begin{document}
% \maketitle

\phantom{a}
\vspace{15mm}
\begin{center}


        \Large{
      
     
        \textbf{STARTPAGE}
  
          \vspace{15mm}
          PEOPLE\\
          MARIE CURIE ACTIONS\\
          \vspace{1cm}
          
          \textbf{International Incoming Fellowship  - IIF}\\
          \textbf{Call: FP7-PEOPLE-IIF-2013}
          \vspace{2cm}                   

          PART B
          \vspace{2cm}

          ``\Acronimo''
        }

  \end{center}
\vspace{1cm}
\newpage 
\tableofcontents

\pagebreak

\section{Research and Technological Quality}
\label{sec:sciTec}

 
\subsection{Research and technological quality, and the interdisciplinary and multidisciplinary aspects of the proposal}
\label{sec:sciTecQuality}

The main objective of this research project is to understand the dynamics of \textbf{Coarsening} in \textbf{Classical} and \textbf{Quantum} many body \textbf{Systems}
(referred to in this proposal by the acronym '\textbf{CCQS}') while they are approaching equilibrium after a quench in a  {parameter}; a particularly important aspect of the nonequilibrium dynamics seen in recent experiments. The focus lies on classical nonequilibrium dynamical systems in a closed field theory with a double well potential, the classical evolution of an open system like coupled micromagnets, and the nonequilibrium dynamics of quantum-quenched dissipative spin heterostructures. A 'quench' is defined as the \textit{diabatic} variation of a thermodynamic or other  {parameter} of the system, such as temperature, mass or chemical potential. In a diabatic variation, the rate at which the quench rate is faster than all the relaxation rates intrinsic to the system, thus allowing for approximating the change as instantaneous. 
Coarsening after a quench constitutes the dynamical process by which the characteristic  {size} of the support of the equilibrium phases grows.

The first part of the proposed project involves the analytical and numerical study of classical quenches in $\phi^4$ theory in  {$d$} 
dimension. The potential in the free energy will be constructed to $4^{th}$ order in the order parameter $\phi$. The evolution of the free-energy landscape with the control parameter driving a phase transition guides the understanding of the post-quench dynamics from, typically, a disordered phase to an ordered phase. If the free energy lacks a cubic term, then phase transitions are of second order, driven by instability at criticality. If the free energy has cubic terms, then phase transitions are of first order, driven by metastability at criticality. The project involves the study of quenches to criticality, as well as sub-critical quenches by investigating the microscopic dynamics of the order parameter. This dynamics is governed by a quasi-Newtonian equation of motion with,  {for example, 
a thermalized ensemble of initial states in a free-energy landscape with a single minimum at zero field}. Other descriptions, such as the field 
large-$\mathcal{N}$ approximation, will also be considered. After an instantaneous quench at $t=0$ the subsequent evolution of the order parameter is performed by integrating the equation of motion with the post-quenched parameters  {(the double-well structure)}. The first question to be explored is whether the system re-thermalizes to a steady state or not. This can be tested via the fluctuation-dissipation theorem, where linear responses to a small external field are compared to the field correlations. Subsequently, the temporal behavior of the equilibrating field will be studied. During the \textit{coarsening} process, space-time correlations allow for the identification of a growing length scale. Domains of equilibrium states are expected to grow with this length scale, and a spatial profile of 'kinks' or  {`domain walls'} that demarcate these regions is expected to provide insights into the coarsening process. Domain walls will be identified, whose gradients are expected to 
drive coarsening. Coarsening is thus expected to drive the nucleation and growth of domains that support the equilibrium phase. Near criticality, divergences of time scales via critical slowing down is also known to occur, and the scaling behavior of critical quenches will also be studied analytically using scaling and renormalization group arguments. In addition, the system is expected to build fractal clusters~\cite{fractal}
and the equilibrium and nonequilibrium contributions are multiplicatively separated. In sub-critical quenches, the asymptotic behavior of the characteristic coarsening scale relative to equilibrium correlations is expected to be governed by the dynamic scaling hypothesis~\cite{dynscal}, where the domain structure is statistically independent of time when lengths are scaled accordingly. The goals of this phase of the project are the profiling of the universal scaling laws described above, as well as the evolution of structure interfaces in the order parameter field through the kinks in the solution.

The next phase of the project will involve studying the dynamics of open classical systems \textit{viz.} systems connected to a thermal reservoir. Classical $\phi^4$ theories of the type discussed above can be linked to closed Ising magnets, and open systems of magnets can be studied by dealing with the interactions between magnetic moments on sub-micrometre length scales. These are governed by competition between the dipolar energy and the exchange energy, the outcome of which governs the long range magnetic order, if any. In such systems, the Landau-Lifshitz-Gilbert (LLG) equation is a key model for describing the nonlinear evolution. The micromagnets in this model are built up from fermions acting under a time-dependent Zeeman field, and the Ehrenfest dynamics therein~\cite{gll:review}, together with a phenomenological damping term that takes into account the saturation of the magnetization. The dynamics of many coupled LLG systems connected to a thermal bath (canonical ensemble at equilibrium) can be 
formulated from here. In the continuum limit, the result is a nonlinear partial differential equation in the spin field. The ensuing dynamics links the physics to the microscopic Hamiltonian structures of spin lattices, including planar XX and XY structures~\cite{laxmanan:xxxy} whose quantum evolution will be studied later. This phase of the project aims to formulate the LLG dynamics after a quench across the magnetic phase transitions, and investigate the transition to equilibration in a manner similar to that described in the paragraph above. The dynamics in these systems have wide interdisciplinary relevance. They have intimate connections with many of the well-known integrable soliton equations, including nonlinear Schr\"odinger and sine-Gordon equations~\cite{laxmanan:xxxy,sinegordon}. The possibility of classical chaos in such systems~\cite{gll:review} leads it to other disciplines, such as power generation and synchronization~\cite{lax13} and inhomogeneous filaments~\cite{lax14}. 

The logical continuation of this project now draws attention to the quantum realm in such systems. At zero temperature, a closed quantum system is said to be in equilibrium when it is at the ground state (or generally speaking, any eigenstate) of the Hamiltonian of the system. At finite temperatures, a system at equilibrium can no longer be found in a single eigenstate, but is delocalized in the Hilbert space over many eigenstates with a thermal probability distribution. A \textit{diabatic quench} causes a  {parameter} to vary rapidly in comparison to the relaxation times of the excitations, both thermal and quantum. The nonequilibrium dynamics of such closed quantum systems after a quench, although an ongoing subject of study,  {are somewhat better understood than their open counterparts}, and involve principles like dephasing~\cite{thermalization} 
and the Eigenstate Thermalization hypothesis~\cite{thermalization,krishrev}, the Kibble-Zurek mechanism~\cite{bikashbabu}, Landau-Zener tunneling~\cite{bikashbabu}, semiclassical mean field dynamics like those obtained from the Gross-Pitaevski equation for Bose Einstein condensates~\cite{colrev} and the Ginzburg Landau equation for superconductivity~\cite{rammer}, diagrammatic perturbation theory~\cite{gorkov, volkov}, and others. The quantum dynamics part of this research project will involve the application of some of the above-mentioned ideas, as well as newer stochastic methods, to \textit{open quantum systems} \textit{viz.} systems that can exchange energy via connections to a reservoir of heat. Away from $T=0$, the pure quantum dynamics has to be weighed by the thermal probabilities of the initial state that is presumed to be in thermal equilibrium. For open systems, such dynamics can be studied by completely specifying the nature of the reservoir, and treat the coupled system + reservoir as a closed 
quantum system, or by treating the reservoir stochastically. The nonequilibrium dynamics, thus formulated, allows the profiling of local quantum correlations and coarse-grained responses whose 
spatial extent described coarsening that is expected to grow towards the steady state. Analytics will involve modeling such systems by \textit{spin chain heterostructures}~\cite{arrachea},  constructed by linking finite or semi-infinite XX or XY spin chains (of various anisotropies) to each other at their ends. This model is analogous to well-established models that describe transport in many body systems, as well as nano and mesoscopic systems~\cite{arrachea,openspin, imry}. The dissipative dynamics of the central chain after a quench can thus be obtained from the Hamiltonian quantum dynamics of the entire system of links. Such systems can be mapped onto p-wave BCS superconducting fermions, and the nonequilibrium Dyson equation can be formulated in a manner similar to~\cite{gorkov, volkov}. Post-quench dynamics can be studied by solving the resultant equations for the local fermion correlations by diagrammatic approximations to the nonequilibrium self-energy, either in the collisionless regime~\cite{volkov,
ncnsd2012}, or in the regime where collisions are rapid enough to have relaxed away and only order parameter dynamics remains~\cite{ncnsd2012}. The steady state properties of such systems have already been investigated in the context of thermal transport by Arrachea \textit{et. al.}~\cite{arrachea}, and can easily be adapted to dynamics farther from equilibrium. In this manner, the slow collision dynamics (in the mean field or with Gaussian fluctuations) can be mapped onto a set of nonlinear Schr\"odinger equations coupled by a self consistent update for the superconducting gap. The scaling laws for the responses in such dynamics will also be obtained.  In the fast collision regime, the dynamics can be approximated to be near equilibrium, with a classical $\phi^4$ approximation in the path integral (see~\cite{colrev} and the references therein). This allows for a connection with the classical problem discussed above, and similar numerical methods can be used. Evaluating the Goldstone modes by a linear 
stability analysis of the dynamics around equilibrium will also be done so as to profile excitations and fluctuations about the mean field.

The approach detailed above, with the full Hamiltonian description of the system and reservoir, comes at the price of numerous analytical and computational difficulties. The final part of this project will deal with alternative formalisms of open quantum systems, where the reduced density matrix is evolved in time by a master equation, allowing for the inclusion of incoherent processes which represent interactions with a reservoir. Such dynamics is regularly studied in quantum optics using the \textbf{Kossakowski-Lindblad equation}, where it can represent absorption or emission from a reservoir~\cite{lindblad}. Here, the regular Liouvillean dynamics of closed quantum density matrices are dressed with Lindblad bath operators which act locally on degrees of freedom near each bath. Although this approach is not universally valid, it is a reasonable starting point to study Heisenberg chains such as the ones being studied in this project~\cite{spinchains:lindblad}. Here, as in the previous paragraph, the ends of 
a finite spin chain are coupled to canonical Lindblad spin operators whose amplitudes are determined by the thermodynamics of the reservoir.  This approach is computationally simpler, and the density matrix for such a mixed system can be solved using DMRG methods. The dynamics of coarsening in Lindblad systems can be solved after a quench in this manner.

Coarsening is a very basic aspect of non-equilibrium quantum dynamics. Understanding it would shine light on a vast landscape of quantum phenomena ranging from the process of defect generation in critical quantum relaxation~\cite{relaxation}, thermalization of a closed many-body quantum system~\cite{krishrev, thermalization}, thermalization and effective temperatures of open quantum systems~\cite{thermopen}, glassy systems~\cite{glassy}, mesoscopic systems~\cite{meso}, to the operation of near future quantum devices like an analog quantum computer~\cite{annealing}. The potential and the target of our project, as well as its methodologies, are thus genuinely interdisciplinary and of very broad interest.
 
\subsection{Appropriateness of research methodology and approach}
\label{sec:research_methodology}

The interdisciplinary nature of the project links it to  a diverse set of topics in theoretical physics, necessitating interaction between scientists from different disciplines. The researcher (Analabha Roy) has already experienced enough in the field of nonequilibrium many body dynamics, and will compile his techniques with the Host Scientist (Leticia F. Cugliandolo), who is a leading figure in the field of classical and quantum nonequilibrium dynamics. In addition, the basic framework for solving the quantum problem has been formulated by Profs G. Lozano and L. Arrachea (Universidad de Buenos Aires, Argentina)~\cite{arrachea, arrachea2}, with whom the host has already interacted, as well as A.A. Aligia (Bariloche Atomic Centre, San Carlos de Bariloche, Argentina). The project involves a joint collaboration with them towards applying their formalism to the dynamics of coarsening after a sudden quench.  {Discussions with D. Rossini (Scuola Normale Superiore di Pisa, Italy) 
and R. Fazio (Scoula Normale Superiore di Pisa \& Center for Quantum Technologies, National University of Singapore) on the time-dependent density matrix renormalization group approach to dissipative spin chains have already been initiated.}
The project will combine  multiple methods and paradigms in physics to characterize the general aspects of coarsening in non-equilibrium dynamics. Numerical methods to be used will constitute a variety of simulation algorithms for solving  large nonlinear differential equations and exact diagonalization methods; the researcher already has sound experience with these methods. The key methods used will constitute execution in parallel processor grids with an atomic decomposition of the spatial grid, followed by ghost cell updates using the Message Passing Interface for each time step. Time steps can be performed by conventional quadrature and finite element methods, or more modern symplectic integrators appropriate for Hamiltonian systems~\cite{symplectic}.   {The onset of chaos in the coupled LLG problem via period-doubling bifurcation is of particular interest}~\cite{gll:review}, and analytical treatments of period doubling bifurcation via renormalization group techniques and the analysis 
of Feigenbaum numbers~\cite{hilborn} will be used. The quantum problem will be tackled both analytically and numerically. Analytical techniques in the quantum realm involve the Keldysh theory of nonequilibrium quantum fields as applied to spin chain heterostructures. Starting from the BCS model that is equivalent to that of the spin chains, fermion correlations can be obtained using diagrammatic approximations to the nonequilibrium self-energy, and solving the Dyson equation for complex contour-ordered Keldysh Green's function after mapping it to the real time Green's functions in the RAK formalism using Langreth's theorem and associated rules~\cite{rammer, arrachea}. Different regimes of interest can be identified by associating them with collision rates, and scaling laws for defects obtained  using modified Landau-Zener theory~\cite{fermidyn}, as well as other methods that are more appropriate for 
impulse quenches (see~\cite{ncnsd2012} and references therein). The physics of dissipation in quantum systems can also be modeled by a second approach using Lindblad operators. The ensuing Kossakowski-Lindblad dynamics  can be simulated numerically with novel approaches to the time evolved block decimation using the Density Matrix Renormalization Group technique, including the implementation of these algorithms in distributed grid computing environments using established parallel-programming paradigms~\cite{white:pdmrg}. The researcher and host plans to collaborate with Profs D. Rossini and R. Fazio, and combine their expertise in this area to tackle this final part.

\subsection{Originality and innovative nature of the project, and relationship to the 'state of the art' of research in the field}
\label{sec:originality}
Nonequilibrium dynamics of many body systems provide deep insights into several complex phenomena in nature, ranging from the behavior around phase transitions in bulk matter, biological systems,  to the creation of the known forces of the universe. In addition, studies in Quantum Annealing indicates that there is a very deep relationship between different aspects of quantum non-equilibrium dynamics and the basic limitations of a quantum computer~\cite{annealing}. Nonequilibrium dynamics of open quantum systems are an important part of quantum optics, quantum measurement theory, quantum statistical mechanics, quantum information science, quantum cosmology and semiclassical approximations~\cite{openq}. 

Due to the many implications and interdisciplinary nature of nonequilibrium dynamics, a comprehensive study of the approach to equilibrium in such systems would provide insights into the behavior of several nonequilibrium systems that are actively studied in academia. Also, the LLG equations are related to the dynamics of several important physical systems such as ferromagnets, vortex filaments, moving space curves, $\sigma$-models in particle physics, the spin torque effect in 
nanoferromagnets in the field of spintronics, and others (see~\cite{gll:review} and references therein). The classical kinetics of systems undergoing critical dynamics or an ordering process is an important problem for condensed matter physicists, since it enhances the generic understanding of phenomena not fully understood, such as pattern formation in nonequilibrium systems and the approach to equilibrium in systems with slow dynamics. In addition, most treatments of open quantum systems out of equilibrium have involved phenomenological approaches using generic structures~\cite{openspin}, restricting  themselves to the equilibrium steady state, and often performed within the linear response regime with small temperature gradients. Therefore, it is desirable to develop alternative approaches designed to treat systems far out of equilibrium while they are coarsening towards steady states. The Keldysh formalism provides such a framework, and a detailed investigation of nonequilibrium dynamics using Keldysh 
methods is sorely needed in order to form a complete understanding of such systems. Finally, the detailed study of quantum nonequilibrium dynamics has, for the most part, been restricted to closed quantum systems, and studies of open systems have begun fairly recently~\cite{daley}. Thus, studies of coarsening in open quantum systems will contribute towards a fledgling area of research. 

\subsection{Timeliness and relevance of the project}
\label{sec:timeliness}
The relevance and importance of this upcoming research field cannot be overstated in view of several theoretical and experimental developments that took place over the last couple of decades, such as spinodal decomposition~\cite{spinodal}, magnetic domain growth in ferromagnets~\cite{puri}, and in the field of ultracold atoms (see~\cite{ultracold, colrev, fermidyn,ncnsd2012} and references therein). The high degree of tunability in ultracold systems allow the rendering of slow dynamics in regimes that are inaccessible in traditional solid state systems. In view of these recent experimental advances allowing for the study of coarsening in quantum quenches, the above project is no mere theoretical investigation with far-fetched implications. All the results that will be determined by this project lie well within experimentally accessible regimes, and can be tested using the diverse array of experimental setups mentioned above. The publications that arise out of this project will propose experimental set-ups in 
cold-atomic systems where the theoretical observations may be
realized and tested. The project would thus closely follow the development of these experimental fields. The quantities that this project aims to evaluate in order to investigate coarsening can be directly measured in the laboratory using both \textit{in situ}, as well as time-of-flight measurements of ultracold atoms. Recent theoretical advances in the dynamics of coarsening are providing deep insights regarding various aspects of critical phenomena, and are particularly relevant in quantum faster information processing and computation. Therefore, the time is ripe for a research project that coordinates all these developments.

\subsection{Host scientific expertise in the field}
\label{sec:host_expertise}

 {
The Host Scientist (Leticia F. Cugliandolo) has worked on the out of equilibrium dynamics of complex classical and quantum 
systems since 1993. Her contributions to the field are manifold. The main ones are the proof that mean-field disordered models~\cite{Cuku} and correctly identified mode-coupling theories~\cite{Bocukume} capture the aging properties of glasses, and the identification and comprehension of 
effective temperatures in slowly relaxing or weakly perturbed glassy systems~\cite{Cukupe}. More recently, she started working on 
the non-equilibrium dynamics of quantum systems and coarsening issues, the main subjects of research in this project. Again, many research 
articles certify her activity in these fields (one relevant example is~\cite{Focuga}) and she has given numerous lectures and seminars
on related problems (see her webpage at \url{http://www.lpthe.jussieu.fr/~leticia}) .
}

\subsection{Quality of the group/supervisors}
\label{sec:gropu_quality}
 {
The host institution, LPTHE, (Laboratoire de Physique Th\'eorique et Hautes Energies, directed by Prof. Olivier Babelon) is located on the campus Jussieu in the $5^{th}$ arrondissement of Paris. LPTHE is one of the major theoretical physics laboratories in France. It is part of Universit\'e Pierre et Marie Curie, Paris VI (Sorbonne), one of the top universities in France~\cite{mcranking}, and associated with the CNRS. It is thus an integral part of the world scientific community, and imbibes a strong scientific temper in their members. More than a third of its members are permanent faculty, a high ratio among theoretical physics departments in France. The members work in many branches of physics, mathematics and other sciences, and their research interests range from  statistical mechanics and condensed matter to particle theoretical physics, string theory and mathematical physics. In particular, LPTHE has a strong focus on statistical physics and condensed matter, and includes a very prominent group of 
physicists with strong citation metrics who work on subjects like conformal field theory, nonequilibrium physics, quantum information and computing, and others. Financial support for research activities is provided by Paris 6 University, CNRS and a number of contracts with  French Agence Nationale de la Recherche (ANR), and other laboratories and agencies in Europe and around the world.  Each year, the LPTHE produces around one hundred papers published in specialized refereed journals with high impact factors and immediacy indices, such as European Journal of Physics, Europhysics Letters, JSTAT: Theory and Experiment, Journal of Physics, the Physical Review Series, JHEP, Gravitation and Cosmology, Physics Letters, and others.  LPTHE is also part of a larger entity, the F\'ed\'eration de Recherche Interactions Fondamentales~\cite{frif}, which brings together several laboratories in Paris via common seminars and visitors, workshops and other scientific activities. The LPTHE organizes general colloquia in 
theoretical physics, laboratory seminars and weekly workshops on statistical mechanics and condensed matter.}

 {
As of August $2013$, the LPTHE counts around $25$ permanent members, and  permanently hosts around $10$ postdoctoral researchers and $10$ PhD students. Being situated in central Paris, these students and post-docs are exposed to the bursting research activity of this as well as all other institutions in the area. The senior members of the statistical physics and condensed matter group in LPTHE are Vol. Dotsenko, B. Dou\c{c}ot, L. Faoro,  L. Ioffe, M. Picco, S. Teber  and the host scientist, L. F. Cugliandolo. A mathematical physics group with, among others, P. Zinn-Justin and J-B Zuber as permanent members, is in close contact with this group. In connection with the host scientist research program, the group regularly hosts invited professors from Europe (e.g. A. Gambassi from SISSA, F. Corberi from  Salerno) or from abroad (e.g. C. Chamon from Boston or G. Lozano from Argentina), and keeps regular collaborations with researcher of other Parisian institutions such as M. Tarzia, F. Zamponi and G. Biroli from 
neighboring LPTMS, LPT-ENS and Saclay. The group organizes weekly specialized seminars as well as journal clubs for students and post-docs. The University also hosts a colloquium. The program of most seminars of the Paris area can be consulted at the 
web site \url{http://semparis.lpthe.jussieu.fr/} hosted at LPTHE. A mirror of arXiv.org is also hosted at  the computer facilities of the 
lab. Training at LPTHE has been very successful in the past. Several former students and post-docs in the field have found permanent positions in France (for instance, R. Santachiara at Orsay or P. Pujol at Lyon and next Toulouse) and abroad (e.g. G. Delfino at SISSA).}

%%%%%%%%%%%%%%%%%%%%%%%%%
\newpage
\section{Transfer of Knowledge}
\label{sec:training}
\subsection{Clarity and quality of the transfer of knowledge objectives}
\label{sec:training_objectives}
The objectives for the transfer of knowledge are
\begin{enumerate}
 \item 
 Transfer of knowledge and expertise in nonequilibrium field theory using Schwinger-Keldysh formalism, dynamical systems and chaos in classical and quantum problems, many body physics and condensed matter theory to the host.
 \item
 Training of graduate students in many body theory and computational methods for complex dynamical many body problems using parallel clusters.
 \item
 Facilitation and initiation of existing and new collaborations between host and several collaborators across multiple institutions, in the European Union and elsewhere, bringing their knowledge into Europe by the researcher and host.
 \item
 Presentation of research results in multiple conferences and symposia in Europe and elsewhere. The researcher plans to present his works in at least $5$ international meetings of particular interest for the dissemination of the research to others within Europe and beyond; the March Meeting of the American Physical Society, the National Conference of Nonlinear Systems and Dynamics, India, the STATPHYS conference of the  International Union of Pure and Applied Physics, the Annual Beg Rohu Summer School on topics in statistical physics and condensed matter in Saint-Pierre-Quiberon, Brittany, France, the IWNET workshops organized by ETH-Zurich, and others.
\end{enumerate}


\subsection{Potential of transferring knowledge to European host and/or bringing knowledge to Europe}
\label{sec:training_relevance}
With the proposed project, the researcher will bring several sorts of unique knowledge to the European Union. First, he will bring his extensive expertise in numerical methods. These were founded during his graduate years studying classical and quantum chaos in the University of Texas. The group of his supervisor, Prof Linda E. Reichl, was one of the few groups in North America studying the implications of classical dynamics in the corresponding quantum system as it is periodically driven. Studying the onset of chaos assisted adiabatic passage in the quantum Floquet problem numerically entails large system sizes that are quickly and embarrassingly parallelizable in multiprocessor grids, and the researcher gained extensive expertise to do so using the computational and teaching resources of the Texas Advanced Computational Center at UT. His expertise has been extended and refined during his postdoctoral years in India, where he worked on several dynamical problems involving nonequilibrium quantum fields and 
ensuing ordinary and partial differential equations, as well as a working knowledge of the Density Matrix Renormalization Group method for solving quantum many body problems. He did so via a diverse range of multiprocessor grids at the SN Bose National Centre for Basic Sciences, Kolkata, as well as the Saha Institute of Nuclear Physics, Kolkata. His knowledge of several paradigms in parallel computing, such as multithreading, message passing, OpenCL and computing using graphical processors, as well as scripting for parallel grid engines, will ensure that numerical work is done in a timely and efficient manner, optimally utilizing the computational resources available to the host in order to solve the problems detailed in this proposal. In addition to research, the researcher plans to hold teaching workshops in parallel computing at the host institution where he will be able to introduce scientific computing in distributed environments to graduate students, fellow postdocs and faculty members new to the field.

Second, the researcher will bring his extensive knowledge of condensed matter theory, many body physics, nonequilibrium field theory, nonlinear dynamics, dynamical systems and chaos to Europe. His knowledge in these areas were developed under the guidance of several leaders in their respective fields in the United States and India, such as Prof L.E. Reichl (The transition to Chaos in classical and quantum systems), Prof J.K. Bhattacharjee (Nonlinear Dynamics), Prof. Krishnendu Sengupta (Nonequilibrium Field Theory), Dr. Arnab Das (Dynamics of many particle systems), Prof. Arti Garg (Condensed Matter Theory) and others. Thus, his theoretical background encompasses numerous topics, such as  Floquet theory, Keldysh theory as applied to the Fermi BCS problem, Bose Einstein Condensates and the time dependent Gross-Pitaevski Bogoliubov equations and the study of excitations therein, and others. All of this knowledge will be brought to the European host organization and group and shared with colleagues, graduate 
students and other postdocs.

Finally, the researcher and host have made plans to enhance existing collaborations with colleagues in order to expedite the research work detailed in this proposal. Collaborators include  Profs. G. Lozano and L. Arrachea (Universidad de Buenos Aires, Argentina), D. Rossini (Scuola Normale Superiore, Pisa, Italy) and R. Fazio (Center for Quantum Technologies, National University of Singapore). These collaborations will help bring the knowledge and expertise gained by the collaborators from outside the European Union into Europe via interactions with the researcher. In addition, the conferences attended by the host and researcher mentioned in the previous subsection will also be attended by most condensed matter and statistical physicists in Europe, and lectures and unofficial interactions therein will further disseminate expertise throughout the continent.

%%%%%%%%%%%%%%%%%%%%%%%%
\newpage
\section{Researcher}
\label{sec:researcher}
%(maximum 7 pages which includes a CV and a list of main achievements)
\subsection{Research experience}
\label{sec:research_experience}
\begin{enumerate}
\item
\textbf{2011 - Present}: CSIR Senior Research Associate (Scientists' Pool Scheme) in the Theoretical Condensed Matter Physics Department of the Saha Institute of Nuclear Physics, Kolkata, India. Have been working on
\begin{itemize}
 \item 
Nonequilibrium dynamics of ultracold Fermi superfluids in the BCS regime. The focus is on collisionless relaxations of response during a  periodic drive~\cite{fermidyn}.
 \item
The route to thermalization and quantum ergodicity in quenched nonintegrable many-particle systems. The specific problems being studied include the Eigenstate Thermalization Hypothesis and the transition to typicality after periodically modulated non-monotonic quantum quenches in the disordered Ising and Kitaev models, as well as the Fermi-Hubbard model.
\item
Investigating the possibility of using variational Monte Carlo methods to study phase transitions in Bose-Hubbard systems in
non-Abelian Rashba-like gauge fields. Related problems of interest include the nonequilibrium Dyson-Keldysh dynamics of Bose gases in optical lattices with synthetic non-Abelian gauge fields.
\end{itemize}
\item
\textbf{2009 - 2011}: Postdoctoral Fellow at the S.N. Bose National Centre for Basic Sciences, Kolkata, India. During this period, the researcher worked with Prof. Jayanta K. Bhattacharjee (currently the director of Harish-Chandra Research Institute, Allahabad, India) on the evolution of matter waves in Fermi-Bose mixtures after a rapid impulse quench in the Ginzburg-Landau-Abrikosov-Gor'kov regime~\cite{colrev}, as well as the nonequilibrium dynamics of ultracold Fermi superfluids. Over this period, the researcher collaborated with Prof. Krishnendu Sengupta (Indian Association for the Cultivation of Science, Kolkata), as well as Prof. Arnab Das (Max Planck Institute, Dresden, Germany \& IACS Kolkata).
\item
\textbf{Summer 2009}: Postdoctoral Associate at the Center for Complex Quantum Systems, the University of Texas at Austin, United States. During this period, the researcher worked with Prof. Linda E. Reichl on the dynamics of quantum chaos and control in periodic optical lattices using both numerical and analytical Floquet Theory, as well as Landau-Zener tunneling. The specific focus was on the STIRAP (Stimulated Raman Adiabatic Passage) problem where the ground state of attractive Bosons in an optical lattice was driven via Raman processes with the intent to populate an excited state~\cite{floquet:oplattice}.
%The dynamics of this drive was strongly affected by quantum chaos that manifested via avoided crossings in the Floquet eigenspectrum
\item
\textbf{2002 - 2009}: Pursued and obtained a Doctor of Philosophy (Ph.D) at the Department of Physics, University of Texas at Austin, United States under the supervision of Prof. Linda E. Reichl. The researcher presented his dissertation on the Dynamics of Quantum Control in Cold-Atom Systems in May, 2009. The research involved quantifying the effects of quantum chaos in the controlled STIRAP dynamics of two bosons in a double well~\cite{floquet:dblwell}. 
%The underlying classical chaos was seen to manifest in the STIRAP dynamics via avoided crossings in the Floquet eigenspectrum.

During this period, the researcher collaborated and closely interacted with many other scientists: Dr. Kyungsun Na (CQS, UT Austin), Prof. Mark Raizen (Dept. of Physics, UT Austin), Prof John Keto (Dept. of Physics, UT Austin), Dr. Artem M. Dudarev (currently at Max Planck Institute, Dresden, Germany), Dr. Benjamin P. Holder (currently at Ryerson University, Toronto, Canada), and others. 

Following his qualification, the researcher was trained in advanced subjects by many prominent scientists: Prof. A. H. MacDonald (Dept. of Physics, UT Austin) on many body theory, Prof. Qian Niu (Dept. of Physics, UT Austin) on advanced solid state physics, Prof. Duane Dicus (Dept. of Physics, UT Austin) on quantum field theory, Prof. Jack Swift (Dept. of Physics, UT Austin) on nonlinear dynamics, and Profs. Bill Barth and Kent Milfeld (TACC, UT Austin) on scientific computing in distributed environments.
\item
\textbf{Summer 2001}: The researcher had worked on the phenomenon of quintessence and the cosmological constant as a visiting student at the Inter-University Center for Astronomy and Astrophysics, Pune, India under the supervision of Prof. Varun Sahni.
\item
\textbf{1999 - 2001}: During his undergraduate, the researcher worked on the problem of deterministic chaos in the Ehrenfest dynamics in quantum double well oscillators under the supervision of Prof. J. K. Bhattacharjee (currently the director of Harish-Chandra Research Institute, Allahabad, India). The researcher published his first paper in a peer reviewed journal during this period~\cite{myfirstpaper}.
\item
\textbf{Undergraduate (Bachelors) and Masters}: Secured $1^{st}$ class (the highest grade) in Bachelors (B.Sc. with Hons. in Physics, 2000) from Jadavpur University, Kolkata, India. Graduated with Masters (M.Sc. in Physics, 2002) from the Indian Institute of Technology, Kanpur, India.
\item
\textbf{Foreign visits during research:}
The researcher visited the Dept. of Physics, Texas A \& M University, College Station, TX, USA in $2007$ to present ongoing research works at the Joint Meeting of the Texas Section of the  American Physical Society. Research work was also presented at the March meetings of the American Physical Society in 2008 and 2009. Postdoctoral research works were presented in the STATPHYS conference in Kolkata, India in 2010. The researcher was invited to present his works at the 2012 National Conference on Nonlinear Systems and Dynamics (NCNSD), held at the Indian Institute of Science Education and Research, Pune, India. The proceedings of this conference evolved into the first published review article by the researcher~\cite{ncnsd2012}.
The researcher also gave an invited departmental seminar in the Department of Physics, University of Hong Kong, in 2013 after being invited to do so by Prof Shizhong Zhang. During this visit, the researcher established the foundation for a collaboration with Prof Zhang involving the study of the equilibrium phase diagram of the Bose Hubbard model in a synthetic non-Abelian gauge field. 
\item
\textbf{Awards/Fellowships:}
\begin{itemize}
 \item 
 Scored $1^{st}$ rank nationwide (India) in the National Eligibility Test for Lectureship held jointly
by the University Grants Commission and the Council of Scientific and Industrial Research
(CSIR-UGC NET) on July $17$, $2012$.
\item
Awarded the Senior Research Associateship (Scientists' Pool Scheme) from the Council of
Scientific and Industrial Research (CSIR), Government of India, in $2011$.
\item
Awarded the 'Dr. D.S. Kothari Postdoctoral Fellowship in Sciences, Medical \& Engineering
Sciences' from the University Grants Commission (UGC) India, in $2011$.
\item
Scored $99.2162$ percentile nationwide ($17^{th}$ rank) in the Joint Entrance Screening Test (JEST-
$2002$) in Physics, $2002$ and scored $99.0000$ percentile nationwide ($9^{th}$ rank) in JEST-$2000$.
\item
Awarded Certificate of Merit by the Indian Association of Physics Teachers (IAPT) for being
placed in the Nationwide Top 1\% in the National Graduates Physics Examination (NGPE),
$1997$.
\end{itemize}
\newpage
\item
\textbf{Invited Talks:}
\begin{itemize}
 \item 
$2013$: Invited departmental seminar on the nonequilibrium dynamics of quenched ultracold Fermi superfluids at the Department of Physics, University of Hong Kong, Hong Kong SAR, China 
\item
$2012$: National Conference on Nonlinear Systems and Dynamics (NCNSD),Indian Institute of Science Education and Research, Pune, Maharashtra, India
\item
$2010$: STATPHYS-Kolkata VII, Saha Institute of Nuclear Physics, Kolkata, West Bengal, India.
\item
$2009$ A.P.S March Meeting, David L. Lawrence Convention Center, Pittsburgh, PA, U.S.A.
Session P17: Semiconducting Qbits I, Lec P17, 15
\item
$2008$ A.P.S March Meeting, Morial Convention Center, New Orleans, LA, U.S.A.
Session D14: Quantum Information Science in AMO, Lec D14 2
\item
Fall $2007$: Joint Meeting, Texas Section A.P.S et al at the Dept. of Physics, Texas A \& M University, College Station, TX, USA
Session B2 AMO1: Atomic, Molecular and Optical Physics, Lec B2.1 
\end{itemize}

\item
\textbf{Invited Visit:}
\begin{itemize}
\item
$2013$: Indian Institute of Science Education and Research, Pune, India. Research group of Prof. G. Ambika. Fully supported by the department.
\item
$2013:$ Bhabha Atomic Research Centre, Mumbai, India. Research group of Prof. S. R. Jain. Supported by the C.S.I.R Scientists' Pool Scheme
\item 
$2013:$ Department of Physics, University of Hong Kong. Research group of Prof. Shizhong Zhang. Fully supported by the department.
\item
$2011:$ Department of Physics, University of Pune, India. Research groups of Prof. B. Dey and Dr. P. Durga Nandini. Fully supported by the department.
\end{itemize}

\end{enumerate}


\subsection{Research results including patents, publications, teaching etc., taking into account the level of experience}
\label{sec:research_results}
\subsubsection{Summary:}
As an undergraduate at Jadavpur University, the researcher worked on the problem of \textbf{deterministic chaos in the Ehrenfest dynamics of quantum double well oscillators} in the semiquantal limit. It was already established by Pattanayak and Schieve~\cite{patsch} that the {semiquantal dynamics} show sensitive dependence on initial conditions due to the contributions of the fluctuations of the quantum operators causing transition to chaos from KAM tori. The researcher conclusively demonstrated, however, that this \textbf{apparent manifestation of classical chaos in a quantum system gets suppressed at much larger time scales}, recovering the full quantum dynamics in that limit~\cite{myfirstpaper}. 

Graduate research works at the Center for Complex Quantum Systems, UT Austin, involved investigating the influence of \textbf{chaos in periodic quantum-controlled dynamics} of mesoscopic systems. The specific dynamics involved \textbf{annealing a confining parameter} by modulated pulses of radiation that resonated with internal excitations. The researcher expanded on works done earlier in the center on periodically driven ultracold atoms in optical lattices using Floquet theory, where level repulsions caused by chaos in the underlying classical dynamics were seen to affect the final outcome of a Stimulated Raman Adiabatic Passage~\cite{stirap}. The researcher \textbf{generalized this theory to interacting systems}, where signatures of the underlying classical chaos could be seen in the quantum phase space generated by the eigenstates of the undriven system, as well as in additional avoided crossings in the Floquet eigenspectrum of the driven system when the periodic drives are tuned to Raman transitions~\
cite{floquet:dblwell}. The work involved \textbf{significant numerical simulations of the Floquet dynamics, as well as that of the underlying classical dynamics}. Theoretical calculations of transition probabilities of avoided crossings in Floquet Space were done using Landau Zener theory. He also gained \textbf{significant expertise in distributed computing during this period}, having been given access to one of the most powerful clusters in the world at the Texas Advanced Computational Center (TACC). This work continued onwards to the first postdoctoral appointment at UT Austin, when the researcher \textbf{contextualized the results obtained for mesoscopic systems to optical lattices with Feshbach resonance-induced attractive Bosons}~\cite{floquet:oplattice}. 

The second postdoctoral research project at SNBNCBS involved many body dynamics that was the opposite of that investigated during his graduate work \textit{i.e.} \textbf{quantum quenching}. In such dynamics, confining parameters are varied \textit{diabatically} at a rate that is significantly more rapid than the internal relaxation times of excitations. The post-quench dynamics of a BCS-BEC mixture in the deep-BEC limit was investigated using \textbf{near-equilibrium approximations to the nonequilibrium Keldysh self-energy} via the Ginzburg-Landau Abrikosov-Gor'kov theory. Sufficiently small quenches are expected to select low-lying Goldstone modes during the ensuing dynamics, where the modes have complex frequencies, causing the coherent matter wave to decay to equilibrium. However, the nonlinear dynamics of the coupled Fermi-Bose mixture showed \textbf{Hopf-like bifurcations} that caused the decaying components of the modes to vanish for small Feshbach detuning, leading to nonlinear mode interference that 
persisted over long time scales. This led the researcher to identify a \textbf{shallow BEC regime} before unitarity where such oscillations produced \textbf{collapse and revival} of the matter wave packet~\cite{colrev}. Such behavior had already been seen in quenched ultracold atom systems, both experimentally~\cite{colrev:expt} and theoretically~\cite{colrev:dicke} in a different context and for highly controlled and system-specific quenches. The researcher showed that \textbf{such macroscopic manifestations occur more generally in ultracold quantum quenches}.

The current works of the researcher at SINP involves looking at \textbf{periodic driving of BCS systems as a quantum quench}, investigating collision-less relaxations of responses in such systems in the Bogoliubov limit. The key results, published in JPCM in $2013$~\cite{fermidyn} are that, \textbf{although single channel BCS systems can be mapped to Ising or Kitaev (depending on pairing symmetry) models via a Jordan Wigner-Fourier transformation, the equivalence is not maintained out of equilibrium} with a periodic drive. The primary contributor to this behavior is the self-consistent BCS gap equation, whose continuance in time converts the decoupled TLS dynamics of Ising/Kitaev systems to \textbf{a strongly coupled system of nonlinear Schr\"odinger equations} in the BCS case. The momentum mixing \textbf{produces a steady state regime that does not respect the symmetry of the drive} near half-filling, and causes qualitative departures from several universal quantum many body dynamical responses seen in 
Ising/Kitaev systems, such as dynamical many body freezing. However, the \textbf{responses  and defect densities follow the same scaling laws as those obtained from Landau-Zener theory} in driven Ising/Kitaev systems, thus \textbf{preserving} other universal behavior such as \textbf{the Kibble-Zurek mechanism} near the quantum critical point for d-wave pairing symmetries. In addition, we have demonstrated that the nodes in the defect density that characteristic of driven Ising models remain preserved with the self-consistent update, thus \textbf{identifying an experimental mechanism for detecting the pairing symmetry of ultracold superfluids}. In addition to the work described above, the researcher was invited by the conveners of the NCNSD conference in Pune, India, to evolve the proceedings of the talk delivered therein into a \textbf{mini-review article} on the nonequilibrium dynamics of ultracold Fermi superfluids. After extensive study of the literature on the subject, the researcher completed the review, emphasizing his contributions in the field, and with \textbf{himself as the sole author}, near the end of $2012$. The review article has since been published in the European Physical Journal as part of the conference proceedings~\cite{ncnsd2012}.

The researcher is currently working on the problem of Eigenstate thermalization in quenched nonintegrable many-particle systems. In particular,  quenching and the route to thermalization in systems such as disordered Ising and Kitaev models, and the Fermi and Bose-Hubbard model. In addition, he is investigating the possibility of using variational Monte Carlo methods to study phase transitions in Bose-Hubbard systems in non-Abelian Rashba-like gauge fields, as well as nonequilibrium dynamics of BCS gases in synthetic non-Abelian gauge fields. He is confident of obtaining publishable results by the end of the calendar year.

\subsubsection{\sc Publications:}
\begin{enumerate}
\item
A. Roy\footnote{Corresponding Author}, '\textit{Nonequilibrium Dynamics of Ultracold Fermi Superfluids}', Invited mini-review (NCNSD $2012$),
Eur. Phys. J. ST, {\bf 222} (3-4), 975-993 (2013).\\
DOI: \url{http://dx.doi.org/10.1140/epjst/e2013-01923-y}\\
arXiv: \url{http://arxiv.org/abs/1211.6936}
\item
A.Roy\footnotemark[\value{footnote}] , R. Dasgupta, S. Modak, A.Das, and K. Sengupta, '\textit{ Periodic dynamics of fermionic superfluids in the bcs regime}',  J. Phys.: Condens. Matter, {\bf 25}, 205703 (2013).\\
DOI: \url{http://dx.doi.org/10.1088/0953-8984/25/20/205703}\\
arXiv: \url{http://arxiv.org/abs/1209.4144}
\item
A. Roy\footnotemark[\value{footnote}] , '\textit{Dynamics of quantum quenching for BCS-BEC systems in the shallow BEC regime}', Eur. Phys. J. {Plus}, {\bf 127}:3, 34 (2012).\\
DOI: \url{http://dx.doi.org/10.1140/epjp/i2012-12034-x} \\
arXiv: \url{http://arxiv.org/abs/1009.2125}
 \item 
A. Roy and L.E. Reichl\footnotemark[\value{footnote}], '\textit{Quantum Control  of Interacting Bosons in Periodic Optical Lattice}',Physica {E}, {\bf 42}, 1627-1632 (2010).\\ DOI: \url{http://dx.doi.org/10.1016/j.physe.2010.01.010}\\
arXiv: \url{http://arxiv.org/abs/0907.3813}
\item
A. Roy and L.E.Reichl\footnotemark[\value{footnote}], '\textit{Coherent Control of Trapped Bosons}', Phys Rev {A} {\bf 77}, 033418 (2008). DOI: \url{http://link.aps.org/doi/10.1103/PhysRevA.77.033418}\\
arXiv: \url{http://arxiv.org/abs/0709.0754}
\item
A. Roy and J.K. Bhattacharjee\footnotemark[\value{footnote}],'\textit{Chaos in the Quantum Double Well Oscillator: The Ehrenfest View Revisited}' Phys. Lett. {A}, {\bf 288}/1-3 (2001).\\
 DOI: \url{http://dx.doi.org/10.1016/S0375-9601(01)00351-6}\\
 arXiv: \url{http://arxiv.org/abs/quant-ph/0108032}
\end{enumerate}
\subsubsection{\sc Papers in Preparation:}
\begin{enumerate}
\item
A. Roy and A. Garg, '\textit{Eigenstate thermalization in drive-quenched quantum systems}'.
\item
A. Roy and A. Das, '\textit{Exotic freezing and thermalization in the disordered Ising model}'.
\item
A. Roy and K. Sengupta, '\textit{Quantum quenching in the Bose Hubbard model with a synthetic non-Abelian gauge field.}'
\end{enumerate}
\subsubsection{\sc Teaching}
\begin{enumerate}
\item
$2013$: The researcher has been invited by Dr. A. Das of the Department of Theoretical Physics, Indian Association for the Cultivation of Science, Kolkata, to host an inaugural teaching workshop on High Performance and Parallel Computing for Scientists and Engineers. The researcher is slated to design a short curriculum that will introduce participants to the basics of High Performance Computing, as well as commonly used paradigms in the same such as OpenMP and MPI. The target audience consists of masters and graduate students who need to code and execute numerics with large problem sizes in a distributed computing environment.
\item
$2007-2008$: Grader at the Department of Astronomy, University of Texas, Austin, U.S.A. During this period, the researcher assisted the primary instructors in Introductory (freshman year) astronomy courses by grading assignments, proctoring examinations and grading final exams.
\item 
$2002-2007$, $2008-2009$: Graduate Teaching Assistant at the Department of Physics, University of Texas, Austin, U.S.A. During this period, the researcher taught several undergraduate labs involving Electromagnetism and Optics, Mechanics and Fluid dynamics. The researcher also participated in the design of the laboratory manual for the Electromagnetism and Optics lab course. The researcher also held tutorials in classical mechanics for freshman and sophomore undergraduates.
\end{enumerate}

\subsection{Independent thinking and leadership qualities}
\label{sec:indep_thinking}
From his undergraduate years, the researcher was made aware of the necessity for independent thinking during his first research project. His collaborator (Prof. J.K. Bhattacharjee), a very prominent figure in nonlinear dynamics research, had encouraged him to grow as an independent scientist. This enabled him to develop himself as a self-dependent researcher in the field of quantum dynamics, allowing him to choose his doctoral research subject beforehand. Independent thinking was also encouraged during his graduate years by his doctoral supervisor (Prof. L.E. Reichl), leading him to continue onwards to his postdoctoral research without any formal supervision. The current fellowship of the researcher, a Senior Research Associateship with the Council of Scientific and Industrial Research in India, gives him considerable freedom to form multiple collaborations in diverse fields (see the Research experience section). The researcher's leadership skills have developed during his teaching appointments as a graduate 
student as well. He has also been invited to inaugurate and teach a workshop on parallel computing for scientists by Dr. A. Das at the Indian Association for the Cultivation of Science, to be held later in $2013$. His leadership skills are expected to develop even further during that time. He is leading collaborations in India (Prof. Krishnendu Sengupta, Dr. Arnab Das, Dr. Arti Garg), and Hong Kong (Prof. Shizhong Zhang), and is confident of a productive collaboration with the host mentioned in this proposal if the fellowship is awarded. This clearly reflects his ability to take leadership in research.


\subsection{Match between the fellow's profile and project}
\label{sec:match}
As an independent and prolific researcher from a very early stage, the applicant had the opportunity to gain considerable expertise in the various numerical methods that are required for this project, ranging from integrable dynamics of quenched and annealed systems, Floquet algorithms, Finite Element methods, data analysis and technical graphics, to exact diagonalization in distributed computing environments, large scale parallel computing, optimization problems, partial differential equations, and Density Matrix Renormalization Group techniques. All of these techniques lie at the heart of the proposed project. In addition, the applicant's graduate studies and graduate/postdoctoral works have endowed him with an exhaustive background in theoretical many body physics and nonlinear dynamics, all of which will prove useful towards the goals set in the proposed project. Of particular importance is his knowledge of the quantum theory of nonequilibrium fields using the Schwinger-Keldysh formalism, which was 
useful 
in his earlier research works in the dynamics of quenched and driven BCS-BEC systems~\cite{colrev,ncnsd2012}.
The researcher has already studied Keldysh theory in BCS systems and applied to his research the nonequilibrium Bogoliubov de-Gennes equations that arise from the Dyson equations in complex time~\cite{fermidyn, ncnsd2012}. Thus, he is equipped to  start exploring theoretical models of dissipative quantum spins as outlines in sections~\ref{sec:sciTecQuality} and~\ref{sec:research_methodology}. He has the right background, maturity and temperament necessary for this novel project. The diversity of his specializations also adds to his suitability for this multi-disciplinary project.
   
\subsection{Potential for reaching a position of professional maturity}
\label{sec:potential_position}
As can be ascertained from his scientific and pedagogical activities, the researcher is quite mature already. With suitable mentoring and further professional training by the host, he would easily attain the desired level of maturity.

\subsection{Potential to acquire new knowledge}
\label{sec:potential_knowledge}
The researcher developed and implemented all the necessary numerical tools required for his research during his graduate years, and has always keeps abreast with published research literature in all related fields. His theoretical background in solid state, nonlinear, quantum and many body physics is extensive and grows rapidly during his reviews of ongoing research works. He has collaborated with many academics, and successfully utilizes his knowledge in independent research. This clearly reflects his ability to acquire new knowledge, assimilate it and apply it towards solving new and exciting research problems.
  
%%%%%%%%%%%%%%%%%%%%%%%%%
\newpage
\section{Implementation}
\label{sec:implementation}


\subsection{Quality of infrastructures/facilities and international
  collaborations of host}
\label{sec:facilities}

 {
The LPTHE has local computer calculus facilities that are sufficient to ensure the 
correct development of the numerical part of the research project. It 
counts with a library with the main Theoretical Physics textbooks
(and the possibility of buying any desired issue). Post-docs and 
visitors have their own office with a table computer from which the cluster can be 
accessed. Common spaces for discussions and 
seminars are also available. In addition to these, LPTHE is located
in the Jussieu campus and access to larger computer facilities
as well as the University Library are possible upon registration. 
}

 {
The host is engaged in many international collaborations, especially with groups in 
Buenos Aires Argentina (Lozano, Arrachea), Rio de Janeiro Brazil (Barci), Porto Alegre Brazil
(Arenzon), Trieste Italia (Gambassi), Salerno Italia (Corberi), Bari Italia (Gonnella), Boston USA
(Chamon), Princeton USA (Aron). She is involved in an IRSES program between France
and Japan (S-I Sasa and H. Hayakawa from Kyoto, H. Nishimori from Tokyo and many other
professor participate in this project) 
and starts hosting mid-term visitors from Japan regularly. She lectures in international 
schools regularly and this allows her and her students to meet researchers in the field.
}

\subsection{Practical arrangements for the implementation and
  management of the project}
\label{sec:impl_management}
% {What should we write here? Just invent a couple of lines saying that 
%you'll have your office space, that the door of my office is always open to start
%discussions, and that the coffee room is just one door away from my office.
%}
 {LPTHE provides all its postdoctoral fellows with excellent facilities and infrastructure. The host scientist will provide office space for the researcher, including supplies and a computer workstation, and will always be available for academic discussion and guidance. The host institution will also provide access to academic journals via online and offline subscriptions, broadband internet for research work and communication, institute and university libraries for books, journals and other reference materials, computational and cluster resources of the university and institute, and licenses for any proprietary software that the researcher may require for his work. In addition to research offices, the LPTHE has a seminar room which regularly hosts local seminars and contains the institute library. The library is well stocked with classic and modern text books and other research material on condensed matter theory, many body and field theory, statistical theory and mathematics, among others. 
Finally, the LPTHE has a common coffee room open to all staff, and is easily accessible from any workspace in the institute.}

\subsection{Feasibility and credibility of the project, including work
  plan}
\label{sec:feasibility}
The work plan for the proposed project consists of \textit{work elements} and the time period during which each element will be implemented. Work elements consists of background research, exploration of methodologies, objectives and milestones that can help assess the progress of the project. The various work elements, their feasibility and credibility, and their time schedule are given below.
\newpage
\textbf{Work Elements: Feasibility and Credibility}
\begin{enumerate}{
\item
Literature search and reading.
\item
Enhanced interaction with collaborators: G. Lozano and L. Arrachea (Universidad de Buenos Aires, Argentina), D. Rossini (Scuola Normale Superiore, Pisa, Italy) and R. Fazio (Center for Quantum Technologies, National University of Singapore). Visit institutes of the collaborators, or invite collaborators to host institute for brief enhanced interaction and formulation of problem.
\item
Setting up and exploring techniques I: Numerical methods for $1-d$ classical scalar $\phi^4$ quench dynamics, designing appropriate algorithms in clusters using Message Passing Interfaces for parallelization. Research on numerical techniques for solving coupled Gilbert Landau-Lifshitz equations. Adapt algorithms from previous $\phi^4$ quenches  accordingly.
\item
Work out problems and write papers I: Run code developed in previous work element for various quenches and other parameters. Consider quenches at or near criticality and away from criticality. Choose appropriate scale for coarse-graining responses and study the time evolution of the same. Characterize domain walls and their evolution. Look at universal scaling properties near criticality and attempt to make analytical approximations to compare with numerical results. Collate results and write papers with host and collaborators for peer-review.
\item
Present previous results at key conferences and symposia. Engage with peer-reviewers and implement suggestions, corrections and modifications, if any.
\item
Setting up and exploring techniques II: Evaluation of exact differential equation of the nonequilibrium dynamics of quantum spin systems from Keldysh theory and designing algorithms for solving the same. Research on optimal implementations of quantum time-evolved block decimation/DMRG methods for stochastic Lindblad systems using parallel processors and design algorithm for quantum XY spin system coupled to a bath and quenched.
\item
Work out problems and write papers II: Run code developed in previous work element for various quenches and other parameters. Similar approach to the classical case. Find coarse graining and profile. Check scaling. Collate and send drafts for peer-review
\item
Present previous results at key conferences and symposia. Engage with peer-reviewers.
\item
Begin considering neighbouring areas and improvements on work done so far.
\item
Implement in more complex systems and problems.
}\end{enumerate}
\newpage
\textbf{Time schedule of activities through Bar Diagram:} \\

See above for key to work elements\\

\begin{tabular}{|c|c|c|c|c|c|c|c|c|c|c|c|c|}
\hline
0 & 2 & 4 & 6 & 8 & 10 & 12 & 14 & 16 & 18 & 20 & 22 & 24 \\
\hline
Month/ & & & & & & & & & & & & \\
Work Element & & & & & & & & & & & &  \\
\hline
1 & \cellcolor[gray]{0} & & & & & & & & & & & \\
\hline
2 & & \cellcolor[gray]{0}& & & & & & & & & &   \\
\hline
3 & & & \cellcolor[gray]{0} &\cellcolor[gray]{0} & & & & & & & & \\
\hline
4 & & & & \cellcolor[gray]{0}& \cellcolor[gray]{0}& & & & & & & \\
\hline
5 & & & & & &\cellcolor[gray]{0} & & & & & & \\
\hline
6 &  &  &  &  &  &\cellcolor[gray]{0}   &\cellcolor[gray]{0}  & \cellcolor[gray]{0} &  &  & &  \\
\hline
7 &  &  &  &  &  &  &  & \cellcolor[gray]{0}& \cellcolor[gray]{0}  &  & &  \\
\hline
8 &  &  &  &  &  &  &  & & &\cellcolor[gray]{0}   &  &  \\
\hline
9 &  &  &  &  &  &  &  & & & \cellcolor[gray]{0} &\cellcolor[gray]{0}  &  \\
\hline
10 &  &  &  &  &  &  &  & & & & \cellcolor[gray]{0}&\cellcolor[gray]{0}   \\
\hline
\end{tabular}



\subsection{Practical and administrative arrangements and support for
  the hosting of the fellow}
\label{sec:impl_practical}

% {Again, I'm not fully sure of what we should write here. We have 
%our own secretarial office that is in contact with the one of the university and 
%does all the papers work that might be useful. I'll try to ask the 
%head of the lab about previous applications and what was filled in here, 
%but I'm afraid he's on holidays.}
 { The host will offer every manner of assistance to ensure that the researcher moves to France as soon as the fellowship negotiations are complete. The scientist in charge and the administrative staff of LPTHE have already established links with the researcher, and will remain in close contact in order to facilitate the transition. 
LPTHE has its own secretarial office and is in contact with its counterpart in the university, as well as the International
Relations Department of the same. The latter most office can assist the host in providing the researcher with a welcome agreement that will expedite the visa process, and forward all relevant documentation to the authorities as needed. The secretaries will also oversee all financial aspects of the project and do any the paperwork necessary for the functioning of the project. The researcher will also be introduced to the relevant organizations for accommodation in the Paris metropolitan area, such as the Science Welcome Association, and will be provided assistance with opening a bank account, obtaining insurance for the researcher and family members, payment of taxes, and other necessities. Finally, the host university offers intensive French classes between $2-7$ hours each week according to the initial knowledge level of the participant (that is tested beforehand). The French consulate in the researchers country of origin offers preliminary language courses that he has agreed to take prior to his move to 
France.}

%%%%%%%%%%%%%%%%%%%%%%%%%%%%%%%%\
\newpage
\section{Impact}
\label{sec:impact}

\subsection{Potential for creating long term collaborations and mutually beneficial co-operation between
Europe and the other Third Country }
\label{sec:impact_potential}
The fellow is of high potential and promise. He comes from the Republic of India, a rapidly developing economic power, one of the few developing countries (together with China, Iran and Brazil) among $31$ nations with $97.5$\% of the world's total scientific productivity~\cite{scimpact}, and a major contributor to the global increase in research seen recently~\cite{indoaglob}. He has already shown excellent leadership abilities, developed during his many years of teaching students, and refined during his postdoctoral years, when he coordinated research projects with his collaborators. He has also developed as an independent researcher under the guidance and mentor ship of leading researchers in the United States and India. With this moderate support, he has aided in the development of nonequilibrium physics in India, coming up with very interesting and substantial results, such as the persistent of certain universal behavior (collapse and revival~\cite{colrev}, the Kibble-Zurek mechanism~\cite{fermidyn}) and 
the destruction of others such as dynamical quantum hysteresis~\cite{fermidyn}. This 
has enabled him to establish himself in an extensive network of Indian academics who work in nonequilibrium physics (both theory and experiment) from India's most prominent universities and research institutes, such as the Indian Association for the Cultivation of Science (Prof K. Sengupta and Dr A. Das), the Tata Institute of Fundamental Research (Prof. R. Sensarma), the Indian Institute of Science Education and Research (Profs G. Ambika,  A. Bhattacharyay,  T. S. Mahesh, and U. D. Rapol), the Saha Institute of Nuclear Physics (Profs A. Garg, B.K. Chakraborty and P.K. Mohanty), the Harish Chandra Research Institute (Profs. J.K. Bhattacharjee and P. Mazumdar), the Indian Institute of Science (Profs D. Sen and V. Shenoy), the Indian Institute of Technology, Kanpur (Prof. A. Dutta), and others.

The fellowship is crucial for the development of the career of the researcher in the appropriate direction. With it, his vocation will certainly ascend in this already well-formulated plan of action, and is expected to  reach a formidable level with all its promises. His immediate requirement is a close collaboration with a leading research group which works directly in his field of interest and is both capable and willing to aid him in taking his research to the next logical level. A combination of mentoring by the host, and active research, would help bring out his best work, shaping him quickly as a mature professional and productive physicist, and help him achieve a high academic profile in the statistical, nonlinear, and condensed matter physics communities of Europe as well as India. This will improve his chances for a permanent appointment in one of the above-mentioned research institutes that form the nodes in India's nonequilibrium physics network. Thus, the researcher will be able to interact with 
his colleagues much more closely, and be able to connect his colleagues in India, as well as the European academics with whom he will work during his fellowship, his European host and all of their collaborators. These connections will help them work together in many potential directions that will arise from the results of the proposed research project. The potential for creating long term academic collaborations is thus very high, and the fellowship will aid the researcher in forming mutually beneficial co-operation between the E.U. and India in the field of nonequilibrium classical and quantum physics.

\subsection{Contribution to European excellence and European competitiveness}
\label{sec:contribution_EU}
The very important field of classical and quantum nonequilibrium dynamics has generated new interest over the last decade, in large part, due to advances in experiments involving ultracold atoms. These highly tunable systems make it possible to study nonequilibrium dynamics in regimes hitherto inaccessible in solid state systems. The early experiments involving nonequilibrium ultracold atoms were carried out in the erstwhile Soviet Union and North America (spurred, in part, by cold war competition), followed by significant developments in Europe. The detailed study of heterostructures in a many body system approaching equilibrium is a subset of the above-mentioned studies. Numerical approaches require computational facilities that were realized early on in the United States, with Europe following closely. It is only relatively recently that distributed computing environments in Europe have approached their counterparts in the United States in system size, Floating Point Operations per second, storage and 
memory capacity, and other performance metrics. So far, the bulk of the computational resources have been devoted to military, financial and biological research. As a result, only a few research groups in Europe, such as that of the host, have forayed into the detailed study of coarsening dynamics in established theoretical many body models. Understanding numerous experiments and physical phenomena in these systems relies heavily on the study of coarsening as the system is quenched, and finding universal behavior in such coarsening will lead to significant gains in the understanding of many body systems of high complexity.

The researcher himself is personally connected and in some cases, contributed to the entire of this development. In addition, the host group is widely regarded as  one of the leaders in the field of non-equilibrium dynamics of both classical and quantum systems. Any advancements in the theoretical understanding of such dynamics will stimulate research in nonequilibrium dynamics in Europe, and potentially take the E.U. to the forefront in this topic. Even a marginal success of the project would certainly lay the foundation stone of new paradigm in European science, where the physics of classical and quantum nonequilibrium systems would be used to understand numerous other complex systems, ranging from biological systems to neural networks and even financial models, attracting scientists from a large spectrum of disciplines and trainings.

\subsection{Impact of the proposed outreach activities}
\label{sec:benefit_outreach}
The researcher already has some experience in outreach activities of sorts during his stints teaching students while in graduate school. The bulk of the students of whom he was in charge were not physics majors (typically engineers and pre-medical students), and the researcher was tasked with making physics experiments sufficiently interesting and engaging to students who were, in essence, laypeople. The researcher hopes for the opportunity to apply the skills gained therein during the realization of this proposed research project. He can aid the host prepare material to deliver when students and the general public visit the group to receive first-hand experience and lectures. He can also aid in the mentor ship of undergraduate and masters thesis students who frequently work in short research projects under the host. The researcher also has extensive skills in the typesetting and markup languages that are used by academics to host lecture material and research content on the internet. Thus, he can assist 
the 
host in preparing said material and releasing them to the general public via the webpages of the host institute. All this will help increase the public profile of the host institute via outreach activities.

%%%%%%%%%%%%%%%%%%%%%%%%
\section{Ethical Issues}
\label{sec:ethical_issues}
The project is neither connected to nor in conflict with any ethical issue of any kind, including
those mentioned in the Marie Curie Actions guideline.

%\newpage
%\begin{center}
%      \large\textbf{Ethical issues table}
%\end{center}
%\begin{table}[!h]
%      \centering
%      \begin{tabularx}{\textwidth}{|l|X|c|c|}
%             \hline
%             \rowcolor{black}&\multicolumn{1}{c|}{\cellcolor{black}\color{white}\textbf{Research on Human Embryo/Foetus}}&\color{white}YES&\color{white}Page\\\hline
%            *& Does the proposed research involve human Embryos? & No&\\\hline
%            *&Does the proposed research involve human Foetal Tissues/Cells?&No&\\\hline
%            *&Does the proposed research involve human Embryonic Stem Cells (hESCs)?&No&\\\hline
%            *&Does the proposed research on human Embryonic Stem Cells involve cells in
%            culture?&No&\\       \hline
%             *&Does the proposed research on Human Embryonic Stem Cells involve the derivati
%             of cells from Embryos?&No&\\\hline
%             &I CONFIRM THAT NONE OF THE ABOVE ISSUES APPLY TO MY PROPOSAL&Yes&\cellcolor{gray}\\\hline
%      \end{tabularx}
%\end{table}
%\begin{table}[!h]
%      \centering
%      \begin{tabularx}{\textwidth}{|l|X|c|c|}
%             \hline
%             \rowcolor{black}&\multicolumn{1}{c|}{\cellcolor{black}\color{white}\textbf{Research on Humans}}&\color{white}YES&\color{white}Page\\\hline
%            *& Does the proposed research involve children?&No &\\\hline
%            *& Does the proposed research involve patients?&No &\\\hline
%            *& Does the proposed research involve persons not able to give consent?&No &\\\hline
%            *& Does the proposed research involve adult healthy volunteers?&No &\\\hline
%            & Does the proposed research involve Human genetic material?&No &\\\hline
%            &  Does the proposed research involve Human biological samples?&No &\\\hline
%            &  Does the proposed research involve Human data collection?&No &\\\hline            
%            &I CONFIRM THAT NONE OF THE ABOVE ISSUES APPLY TO MY PROPOSAL&Yes&\cellcolor{gray}\\\hline
%      \end{tabularx}
%\end{table}
%\begin{table}[!h]
%      \centering
%      \begin{tabularx}{\textwidth}{|l|X|c|c|}
%            \hline
%            \rowcolor{black}&\multicolumn{1}{c|}{\cellcolor{black}\color{white}\textbf{Privacy}}&\color{white}YES&\color{white}Page\\\hline
%            & Does the proposed research involve processing of genetic
%            information or personal data (e.g. health, sexual
%            lifestyle, ethnicity, political opinion, religious or
%            philosophical conviction)?&No &\\\hline
%            
%            & Does the proposed research involve tracking the location or observation of people? &No & \\\hline
%
%            &I CONFIRM THAT NONE OF THE ABOVE ISSUES APPLY TO MY
%            PROPOSAL&Yes&\cellcolor{gray}\\\hline
%      \end{tabularx}
%\end{table}
%\begin{table}[!h]
%      \centering
%      \begin{tabularx}{\textwidth}{|l|X|c|c|}
%            \hline
%            \rowcolor{black}&\multicolumn{1}{c|}{\cellcolor{black}\color{white}\textbf{Research on Animals}}&\color{white}YES&\color{white}Page\\\hline
%
%            & Does the proposed research involve research on animals?&No &\\\hline
%&  Are those animals transgenic small laboratory animals?&No &\\\hline
%&  Are those animals transgenic farm animals?&No &\\\hline
%*& Are those animals non-human primates?&No &\\\hline
%  &Are those animals cloned farm animals?&No &\\\hline
%  &I CONFIRM THAT NONE OF THE ABOVE ISSUES APPLY TO MY PROPOSAL&Yes&\cellcolor{gray}\\\hline
% \end{tabularx}
%\end{table}
%\begin{table}[!h]
%      \centering
%      \begin{tabularx}{\textwidth}{|l|X|c|c|}
%            \hline
%            \rowcolor{black}&\multicolumn{1}{c|}{\cellcolor{black}\color{white}\textbf{Research
%                Involving Developing
%                Countries}}&\color{white}YES&\color{white}Page\\\hline
%
%            &Does the proposed research involve the use of local resources (genetic, animal,
%            plant, etc)?&No &\\\hline
%            
%            &Is the proposed research of benefit to local communities (e.g. capacity building,
%            access to healthcare, education, etc)?&No &\\\hline
%            
%            &I CONFIRM THAT NONE OF THE ABOVE ISSUES APPLY TO MY
%            PROPOSAL&Yes&\cellcolor{gray}\\\hline
%      \end{tabularx}
%\end{table}
%
% \begin{table}[!h]
%      \centering
%      \begin{tabularx}{\textwidth}{|l|X|c|c|}
%            \hline
%            \rowcolor{black}&\multicolumn{1}{c|}{\cellcolor{black}\color{white}\textbf{Dual Use}}&\color{white}YES&\color{white}Page\\\hline
%
%            &Research having direct military use&No &\\\hline
%
%            &Research having the potential for terrorist abuse&No
%            &\\\hline
%
%            &I CONFIRM THAT NONE OF THE ABOVE ISSUES APPLY TO MY
%            PROPOSAL&Yes&\cellcolor{gray}\\\hline
%      \end{tabularx}
%\end{table}
%
%\newpage

      
%\bibliographystyle{IEEEtranSA}
%\bibliography{}
\begin{thebibliography}{}

\bibitem{fractal}
J.D. Gunton, M. San Miguel, and P.S. Sahni, in \textit{Phase Transitions and Critical Phenomena}, C. Domb
and J.L. Lebowitz eds. (Academic Press, New York, 1983)  {\bf 8} 267; H. Furukawa, Adv. Phys. {\bf 6}, 703 (1985);
J. Langer, in \textit{Solids Far From Equilibrium}, C. Godr'eche ed. (Cambridge University Press, Cambridge, 1992).

\bibitem{dynscal}
B.I. Halperin and P.C. Hohenberg, Phys. Rev. {\bf 177}:2, 952 (1969).

\bibitem{gll:review}
M. Lakshmanan,  Phil. Trans. R. Soc. A {\bf 369}:1939 1280-1300 (2011).

\bibitem{laxmanan:xxxy}
M. Lakshmanan and A. Saxena, Physica D {\bf 237} 885-897 (2008); J. A. G. Roberts and C.J. Thompson, J. Phys. A {\bf 21} 1769-1780 (1988).

\bibitem{sinegordon}
M. Daniel and L. Kavitha, Phys. Rev. B {\bf 66} 184433 (2002); H.J. Mikeska and M. Steiner, Adv. Phys. {\bf40}:3 191-356 (1991). 

\bibitem{lax13}
J. Grollier, V. Cross  and A. Fert, Phys. Rev. B {\bf 73} 060409 (2006).

\bibitem{lax14}
Y.B. Bazaliy, B.A. Jones and S.C. Zhang, Phys. Rev. B {\bf 69} 094421 (2004).

\bibitem{thermalization}
M. Rigol, V. Dunjko, and M. Olshanii, Nature 452, 854 (2008).

\bibitem{krishrev}
A. Polkovnikov, K. Sengupta, A. Silva, M. Vengalattore, Rev. Mod. Phys. \textbf{83}, 863 (2011).

\bibitem{bikashbabu}
A. Dutta, U. Divakaran, D. Sen, B.K. Chakrabarti, T.F. Rosenbaum, G. Aeppli, arXiv:1012:0653 (unpublished).

\bibitem{colrev}
A. Roy, Eur. Phys. J. {Plus}, {\bf 127}:3, 34 (2012).

\bibitem{rammer}
J. Rammer, \textit{Quantum Field Theory of Non-equilibrium States} (Cambridge University Press, Cambridge 2007).

\bibitem{gorkov}
L.P. Gorkov, G.M. Eliashberg, Sov. Phys. JETP \textbf{27}, 328 (1968).

\bibitem{volkov}
A.F. Volkov, Sh.M. Kogan, Sov. Phys. JETP \textbf{38}(5), 1018 (1974).

\bibitem{arrachea}
L. Arrachea, G. S. Lozano, and A. A. Aligia, Phys. Rev. B {\bf 80}, 014425 (2009).

\bibitem{openspin}
M. Michel, O. Hess,  H. Wichterich and J. Gemmer, Phys. Rev. B {\bf 77}, 104303 (2008) 

\bibitem{imry}
Y. Imry, \textit{Introduction to Mesoscopic Physics}, (Oxford University Press, 1997).

\bibitem{ncnsd2012}
A. Roy, \textit{Nonequilibrium Dynamics of Ultracold Fermi Superfluids}, Invited mini-review (NCNSD $2012$),
Eur. Phys. J. ST, {\bf 222} (3-4), 975-993 (2013).

\bibitem{lindblad}
A. Kossakowski, Rep. Math. Phys. {\bf 3} 247 (1972); G. Lindblad , Commun. Math. Phys. {\bf 48} 119 (1976).

\bibitem{spinchains:lindblad}
S. Clark, J. Prior, M. J. Hartmann, D. Jaksch, and M. B. Plenio, New J. Phys. {\bf 12}, 025005 (2010).

\bibitem{relaxation}
W. H. Zurek, U. Dorner, and P. Zoller, Phys. Rev. Lett. {\bf 95}, 105701 (2005).

\bibitem{thermopen}
A. Caso, L. Arrachea, G. S. Lozano, Eur. Phys, J B, {\bf 85}:266, (2012).

\bibitem{glassy}
L. F. Cugliandolo and J. Kurchan, Phys. Rev. Lett. {\bf 71}, 173-176 (1993) 

\bibitem{meso}
L. Arrachea and L. F. Cugliandolo, Europhys. Lett. 70 642 (2005).

\bibitem{annealing}
Arnab Das and B. K. Chakrabarti Eds., \textit{Quantum Annealing and Related Optimization Methods}, Lecture Note in Physics, {\bf 679}, Springer-Verlag, Heidelberg (2005).

\bibitem{arrachea2}
L. Arrachea, Phys. Rev. B {\bf 79}, 104513 (2009) 

\bibitem{symplectic}
E. Forest and R.D. Ruth, Physica D {\bf 43} 105 (1990); H. Yoshida, Phys. Lett. A {\bf 150} (5-7):262 (1990).

\bibitem{hilborn}
R. Hilborn, \textit{Chaos and Nonlinear Dynamics: An Introduction for Scientists and Engineers}, (Oxford University Press, USA, 2001).

\bibitem{fermidyn}
A.Roy, R. Dasgupta, S. Modak, A.Das, and K. Sengupta,  J. Phys.: Condens. Matter, {\bf 25}, 205703 (2013).

\bibitem{white:pdmrg}
E. M. Stoudenmire and S. R. White, Phys. Rev. B {\bf 87}, 155137 (2013).

\bibitem{openq}
Breuer, Heinz-Peter; F. Petruccione. \textit{The Theory of Open Quantum Systems}, (Oxford University Press 2007).

\bibitem{daley}
W. Yi, S. Diehl, A. J. Daley and P. Zoller, New J. Phys. {\bf 14}, 055002 (2012).

\bibitem{spinodal}
A. Sicilia, Y. Sarrazin, J.J. Arenzon, A.J. Bray, L.F. Cugliandolo, Phys. Rev. E {\bf 80} 031121 (2009).

\bibitem{puri}
S. Puri, \textit{Kinetics of Phase Transitions}, S. Puri and V. Wadhawan eds. (CRC Press, Boca Raton 2009).

\bibitem{ultracold}
M. Lewenstein, A. Sanpera, V. Ahufinger, B. Damski, A. Sen De, U. Sen, Adv. Phys. {\bf 56}
243 (2007).

\bibitem{Cuku} 
L. F. Cugliandolo and J. Kurchan,  Phys. Rev. Lett. {\bf 71}, 173-176 (1993).

\bibitem{Bocukume}
J-P Bouchaud, L. F. Cugliandolo, J. Kurchan, and M. M\'ezard, Physica A {\bf 226}:3-4, 243-273 (1996).

\bibitem{Cukupe}
L. F. Cugliandolo, J. Kurchan, and L. Peliti,  Phys. Rev. E {\bf 55}, 3898-3914 (1997). 

\bibitem{Focuga}
L. Foini, L. F. Cugliandolo, and A. Gambassi, Phys. Rev. B {\bf 84}, 212404 (2011).

\bibitem{mcranking}
QS rankings for the Universit\'e Pierre et Marie Curie can be found at \url{http://www.topuniversities.com/universities/universit\%C3\%A9-pierre-et-marie-curie-upmc}

\bibitem{frif}
The F\'ed\'eration de Recherche Interactions Fondamentales (FRIF) combines several laboratories around a common project and brings together three laboratories in Paris \textit{viz.} the LPNHE, the LPT-ENS and LPTHE . It focuses on theoretical physics and experimental particle physics and astroparticle physics. Details can be found at  \url{http://www.lpthe.jussieu.fr/fed/}.

\bibitem{floquet:oplattice} 
A. Roy and L.E. Reichl,  Physica {E}, {\bf 42}, 1627-1632 (2010). 

\bibitem{floquet:dblwell}
A. Roy and L.E. Reichl,  Phys. Rev. {A} {\bf 77}, 033418 (2008).

\bibitem{myfirstpaper}
A. Roy and J.K. Bhattacharjee, Phys. Lett. {A}, {\bf 288}/1-3 (2001).

\bibitem{patsch}
A.K. Pattanayak and W.C. Schieve, Phys. Rev. Lett. {\bf 72}, 2855 (1994).

\bibitem{stirap}
K. Na and L.E. Reichl, Phys. Rev. A {\bf 70}, 063405 (2004); K. Na and L.E. Reichl, Phys. Rev. A {\bf 72}, 013402 (2005); B. P. Holder and L. E. Reichl, Phys. Rev. A {\bf 72}, 043408 (2005).

\bibitem{colrev:expt}
M. Greiner, O. Mandel, T. W. H\"ansch, and I. Bloch, Nature, {\bf 419}:51-54 (2002).

\bibitem{colrev:dicke}
H.B. Huang, C.X. Yang, L.J. Sun, L. Chen, and J. Li, Physics Letters A, {\bf 372} (36):5748-5753 (2008).


\bibitem{scimpact}
D.A. King, Nature {\bf 430}, 311-316 (2004). Also see D. Dickson, \textit{China, Brazil and India lead southern science output}, SciDev.Net ($16/04/2004$), link: \url{http://www.scidev.net/global/policy/news/china-brazil-and-india-lead-southern-science-outp.html}.

\bibitem{indoaglob}
The Royal Society, \textit{Knowledge, networks and nations, Final report}, RS Policy document 03/11
Issued: March 2011 DES2096, ISBN: $9780854038909$, Available online at \url{http://royalsociety.org/policy/projects/knowledge-networks-nations/report/}.
\end{thebibliography}

\phantom{a}

\newpage
\phantom{a}
\vspace{15mm}
\begin{center}


        \Large{
      
     
        \textbf{ENDPAGE}
  
          \vspace{15mm}
          PEOPLE\\
          MARIE CURIE ACTIONS\\
          \vspace{1cm}
          
          \textbf{Intra-European Fellowships - IEF}\\
          \textbf{Call: FP7-PEOPLE-IEF-2013}
          \vspace{2cm}                   

          PART B
          \vspace{2cm}

          ``\Acronimo{}''
        }

  \end{center}
\vspace{1cm}


\end{document}
