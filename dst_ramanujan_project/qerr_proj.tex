\documentclass[a4paper,9pt]{article}

%% A template for the IEF Marie Curie action
%
% All very simple code, and standard packages. The bibliography uses
% IEEEtranSA style, which is very similar to alpha.
%
% The ethical issues tables in section B6 could be forced in place
% more elegantly, but it worked for me.
%
% DOUBLE CHECK the details of your call before using this
% template. The name of the sections and subsections changes from call
% to call, and new sections are added and removed.
%
% August 2010, v1.0 - Jesus Nuevo-Chiquero.
%
% This file is provided AS IS, with absolutely no warranty of
% anything. You are welcome to use it, but you assume all risks.
\usepackage[defaultsans]{droidsans}
\renewcommand*\familydefault{\sfdefault} %% Only if the base font of the document is to be typewriter style
\usepackage[latin1]{inputenc}
\usepackage[margin=0.5in]{geometry}
\usepackage{setspace}
\usepackage[numbers, comma, sort&compress]{natbib}
\usepackage{hyperref}

\let\oldthebibliography=\thebibliography
  \let\endoldthebibliography=\endthebibliography
  \renewenvironment{thebibliography}[1]{%
    \begin{oldthebibliography}{#1}%
      \setlength{\parskip}{0ex}%
      \setlength{\itemsep}{0ex}%
  }%
  {%
    \end{oldthebibliography}%
  }

 \title{Research Proposal:\\ Quantum Error Correction from Periodically Driven Nonequilibrium Dynamics }
 \author{Analabha Roy\\Postdoctoral Fellow\\National Institute for Theoretical Physics (NiTheP), Wallenberg Research Centre\\10 Marais Street, Stellenbosch, South Africa}
 \date{\today}

\begin{document}
 \maketitle
\section{Non-Technical Summary}
The primary goal of this project is to minimize via external control the occurrence of Quantum Error in Quantum Information Science and Computing. Quantum error correction is used in quantum computing to protect quantum information from errors due to disorder-dynamics, decoherence and other types of noise. Quantum computers make direct use of phenomena arising from quantum physics, such as linear superposition and quantum entanglement, to perform logical instructions on datasets. These differ from classical computers where instructions and data are encoded into bits that are always in one of two states. In quantum computers, a single quantum bit (qubit) can be a superposition of two states, and can be used to store significantly more information that its classical counterpart. Quantum computing is a fledgling subject, although experiments have been carried out on small qubit systems.

Governments and military all over the world support research into this field, as does the private sector. Their eventual goal is to develop fault-tolerant quantum computation, enhancing the power of computers by many orders of magnitude leading to a unprecedented revolution in computers. Examples include quantum cryptography and cryptanalysis, inherent parallelism (allowing quantum computers to perform millions of computations simultaneously), powerful simulators for complex systems (maybe even the early universe itself), and even emergent artificial intelligence. All these things motivate our study of quantum error correction.  The onset of errors during logical operations in qubits present a major challenge in quantum computing. These errors manifest themselves in the stored information as unwanted flips in specific qubits. Classical bit errors are normally reduced by redundancy \textit{i.e.} making copies of the data and detecting errors by comparison and majority vote. This is impossible in quantum computing due to the no-cloning theorem, which does not permit exact copies of the quantum state.  A key quantity of interest in systems of many interacting quantum bits is one of Entanglement. Entangled states are quantum states superposed from the elementary constituents of a many body system in such a way that the quantum state of each component cannot be described independently of the others. Entangling qubits with a known many body state are believed to reduce quantum errors.  Thus, the generation and control of entangled qubit states is of enormous relevance to quantum computing. However, the onset of Nonequilibrium Dynamics in many qubit systems may cause information errors to propagate in time.

Nonequilibrium dynamics - the time-evolution of unstable states that disallow static thermodynamic descriptions of macroscopic physical systems, are of tremendous interest to condensed matter physicists and statistical mechanics people. Nonequilibrium dynamics is also intimately connected to quantum information science and computing. All quantum computing operations involve inputting a many qubit state through a series of quantum logic gates, all of which are complex interacting many body systems described by Hamiltonians that evolve the qubit (or, equivalently,  the associated observables - depending on the dynamical picture being used) in time. Thus, any computational path must necessarily be described by nonequilibrium dynamics. A recently observed phenomenon in this area, that of Dynamical Many Body Freezing, causes prepared quantum states to freeze to their equilibrium values if the system is periodically driven at a known resonance. This occurs due to Quantum Interference, a linear superposition of quantum states that can cause responses to grow and fall in time, much like the bow waves from the wakes of many different boats moving in parallel criss-cross over water. A great many such interfering terms can cancel out all time evolution (the crest of one wave falling on the trough of another), and lead to freezing (calm waters). In addition, periodically driving a many body system such as a quantum logic gate, can lead to dynamics that can be mapped via Floquet Theory and the Theory of the Renormalization Group, to that of a gate with a different logical outcome. Controlling the nature of the drive (such as a harmonic drive or other time periodic functions) can lead to changes in the transition amplitudes of the underlying gate levels that can be used to direct, in real time, the outcome of the logical operation on the input qubit state. The main goal of this project is to investigate if the phenomenologies associated with periodically driven quantum magnets can be used to control and direct the outcome of quantum logic gates, thereby allowing the possibility of eliminating the degradation of quantum information, either due to isolated bit flips or due to decoherence induced by thermalization or coupling to the external environment. 

The realization of quantum error correction in the manner described above is not obvious currently, since entangled qubit states may have unusual responses to driven dynamics that have not been studied so far. The  goals of this project involve the study of complex many body logic gates realized by interacting quantum magnets. Each magnet, capable of storing a single qubit, interacts with its neighbours via amplitude hopping, as well as repulsive pseudopotentials. These interactions can be modeled in various ways, and our focus will be on one dimensional Ising-like spins with nearest neighbour as well as long range interactions. 

Our studies will require the use of extensive analytical and numerical techniques which are described in the sections below. My own interest in nonequilibrium dynamics stems from my undergraduate years, and I have been working with great passion in this field throughout my graduate and postdoctoral years. I have worked with many luminaries in this area, both in India as well as in the United States, and have developed an extensive network of academics who work in nonequilibrium dynamics. The opportunity offered by this fellowship will give me the ability to achieve the goals described above much more efficiently. I will present the results in national and international conferences in the BRICS countries, Europe, North America and East-Southeast Asia. I should also be able to host frequent academic visits of my colleagues to my host institute, as well as get funding for academic visits of my own to them. I will deliver seminars that will increase interest in this area among undergraduate and graduate students, attracting potential research collaborators.  Given the importance that many academics, as well as people in the industry and government, attach to the prospects of quantum computing, this research project will help keep my host at the final frontier of modern technology.

\section{Ongoing Postdoctoral Research (2013-Present):}

My primary research work has been on the nonequilibrium dynamics of many-particle systems, and the onset of universal behavior therein.
My collaborators and I have been giving special focus to quantum magnets, specifically those that can be modeled by Heisenberg, Ising or Curie-Weiss models. The universal features that I am focusing on are the onset of quantum equilibration and ergodicity via the Eigenstate Thermalization  Hypothesis (ETH), the onset of exotic freezing via the coherent destruction of tunneling, as well as the onset of Many Body Localization (MBL) in the presence of disorder. The first project on the driven disordered Ising model, is being done in collaboration with Prof Arnab Das of the Indian Association of Cultivation of Science, Kolkata. This work is currently being extended to investigate the onset of MBL, as well as the competition between MBL phase transitions and Many Body Freezing, in nonintegrable disordered systems. The second project on the propagation of quantum correlations and entanglement in long range spin magnetism, is being done in collaboration with Dr Lorenzo Pucci of NiTheP, Stellenbosch, and  under the supervision of Prof Michael Kastner at NiTheP. We are also cultivating a collaboration in this area with Professor Kaden R. Hazzard of the Department of Physics, Rice University, Houston USA. Our ongoing focus lies on  the analysis of quantum quenches in long range Ising models by approximate numerical methods, such as the discrete truncated Wigner approximation (DTWA), the quantum BBGKY hierarchy, and combinations therein.

\subsection{Periodically driven disordered spin models:}
\label{subsec:isingrand}
One of the models that I am currently studying is the disordered Edwards-Anderson model out of equilibrium, specifically on the Bogoliubov formalism in the rotating wave limit.  I am attempting an understanding of  the dynamics near the points of ordered freezing, in the spirit of similar works on the impulse quenches of such models, I have performed numerical and analytical studies on a time periodic version, where the disorder is added to the Ising spin system. The current focus is on a regime where the ordered model shows dynamical many body freezing, caused by an exotic feature of periodically driven  quantum dynamics called the coherent destruction of tunneling that has been reported many times over the course of the  $20^{th}$ century in the context of atomic and molecular physics. Here, the evolution of certain simple quantum systems can be strongly suppressed by arranging massive coherent cancellation of transition amplitudes through appropriate periodic drive. I have shown that such dynamical localization can have dramatic manifestations even in a system with extensive number of random interactions. In a disordered quantum Ising chain, I have seen that the timescale of
magnetization decoherence can be enhanced by orders of magnitude by employing a periodic drive with specific values of drive parameters regardless of the initial state. My results can readily be translated to describe similar freezing in a much larger family of fermionic and bosonic systems. I have successfully analyzed the system using asymptotic RG theory, the work has been reviewed and has been published. We are currently in the process of applying the Wilson Numerical Renormalization Group in the Floquet-Hilbert space of time-periodic Hamiltonians to systems that are built from the Edwards-Anderson model with integrability breaking terms, such as the Aubre-Andre model, as well as Ising spin models with dual transverse fields.

\subsection{Kinetic Theory of Long Range Systems:}
Our studies in this area currently focus on  the time evolution of one and two-point correlation functions for long-range interacting quantum spin models on a 1-d ring lattice with three different numerical methods. In the paradigm of the quantum Bogoliubov-Born-Green-Kirkwood-Yvonne (BBGKY) hierarchy we adopt the cluster expansion and truncate the hierarchy by neglecting the third order correlations. We have obtained very rich non-relaxing phenomenology related to a varied choice of parameters defining the Hamiltonian. We have also seen that
equilibration is never obtained within the BBGKY analysis, with the averaged $\sigma^x$ oscillating around zero or in a polarized state. We have compared the BBGKY method with a slow-fast (SF) paradigm which leads to relaxation in all the cases analyzed and to the exact time-scaling in the Ising case, and with the discrete Truncate Wigner Approximation (dTWA), a recently introduced numerical phase-space method. Both the SF and dTWA show relaxation whose time scales in the same way as the oscillation period individuated by the BBGKY analysis. However, the BBGKY dynamics is highly sensitive to bifurcations in the dynamics, and does not always evolve to the expected nonequilibrium steady state. Our ongoing objective is to overcome this limitation and strengthen our methodology by replacing the classical evolution of the Weyl symbol in the dTWA with the truncated BBGKY dynamics of the quantum observable. This will provide us with a powerful and computationally scalable technique with which to investigate the propagation of quantum correlations in a wide range of spin systems, including systems of interacting quantum bits.

\section{Technical Details of the Project}
The main objective of this research project is to understand the onset and propagation of Quantum Errors in many particle systems from the perspective of nonequilibrium many body physics. Quantum errors can arise due to the degradation of the physical information in the state of a quantum system as it is evolved in time. A coherent quantum state can be used to store substantially more information than a system of classical bits of comparable size. Unlike discrete classical 0,1 digital states, a quantum bit (qubit) is continuous, describable by a superposition of states that can be characterized by a continuous parameter in the range [0,1], such as the direction on a Bloch sphere. The propagation of quantum error can be minimized by spreading the information in one qubit onto a highly-entangled state of several (physical) qubits.   However, a closed system of such qubits, evolving in time by the Schroedinger equation, will affect the stored coherent information. At  T=0, if the many body state is an eigenstate of the Hamiltonian of the system, then, in a time-independent Hamiltonian, only phase-evolution occurs . However, the advent of disorder and open links to external reservoirs means that the system is never in a true eigenstate, and therefore has components in the true eigenstates. Each component evolves in time with phase. The study of such dynamics is part of the field of quantum Nonequilibrium Dynamics - the description of quantum systems that can no longer be completely characterized by macroscopic time-independent quantities. When out of equilibrium in a time-independent Hamiltonian, the interference between the time-evolving eigenphases can cause decoherence and loss of information, manifesting themselves as quantum errors. In nonintegrable systems, the phenomenon of ergodicity can cause potentially catastrophic loss of information, since ergodicity leads to an equal a-priory probability distribution in the constituent eigenstates no matter what the initial state was. Other kinds of quantum errors may arise due to quantum shot noise, as well as absorption by the environment in open quantum systems. Therefore, the study and minimization of such errors is of key interest.

The dynamics of quantum systems out of equilibrium has attracted a lot of theoretical and experimental attention in recent days. However, theoretical descriptions are hindered by the fact that little is understood about the dynamics of even the simplest of the quantum systems subjected to the simplest kinds of time-dependent drives. For example, the exact dynamics of a two-level system (such as a single qubit) subjected to a sinusoidal drive is still an unsolved problem, inspiring innovative approximations leading to only a partial understanding. Within many such simple setups, quantum interference can manifest itself through surprising effects which are quite counter-intuitive and unexpected. One such phenomenon, that of Coherent Destruction of Tunneling , can be used to halt the evolution of a simple system by quantum dynamical interference. A many-body version of this phenomenon, known as Dynamical Many Body Freezing, has recently been observed both theoretically and experimentally. Here, 
the evolution of certain simple quantum systems can be strongly suppressed by arranging massive coherent cancellations of transition amplitudes through an appropriate periodic drive.  Naively speaking, such dynamical many-body freezing (as has been observed so far) is likely to be seen only in systems with simple underlying Hamiltonians. This is because the massive coherent cancellations of transition amplitudes that arise due to quantum interference seems possible only by fine-tuning a few parameters of a simple Hamiltonian. However, recent theoretical research suggests that a disordered many body system (which can be described theoretically by a very large number of random parameters) may also show freezing for sufficiently rapid drives while kept at a resonant condition. Note that this phenomenon is distinct from classical dynamical hysteresis, or Anderson localization in disordered systems. Thus, there is an exciting possibility that the evolution of quantum errors in a many body entangled state may be frozen at the desired value, eliminating this type of quantum error altogether!

The primary purpose of this project is to investigate the possibility that a multipartite entangled state may be completely 'frozen' in a manner described in the previous paragraph by applying a periodic drive at a known resonance condition. So far, theoretical research on freezing has focused on the evolution of driven systems of quantum magnets (where entangled states can be prepared), especially on Ising and Kitaev models with nearest-neighbour interactions in 1 and 2 dimensions respectively. Recent research has also drawn attention to this phenomenon in ultracold optical lattices, specifically the Bose-Hubbard model. However, only the time evolution of local responses, such as magnetization or particle density, have been studied and determined to be frozen at resonance. The evolution of non-local quantities, such as correlations or entanglement, is still an open problem. In this project, we plan to start from theoretical models that describe disordered quantum spin structures, such as the disordered 
Ising model (Edwards-Anderson model) in 1D, as well as generalizations to  models with interactions and transverse fields that break internal integrability. The results can easily be carried over to the context of several other systems, e.g., those of hardcore bosons/free fermions in disordered potentials. 

The most powerful analytical techniques in our possession that can be used to understand such dynamics involve Floquet Theory, which can map the dynamics of time periodic Hamiltonians to an effective dynamics of a time-independent Hamiltonian, and the Wilson Numerical Renormalization Group technique. The latter provides tools for constructing controlled approximations of the Floquet Hamiltonian in the large frequency limit using the equivalence between the time dependence of a Hamiltonian and an interaction in its Floquet operator. This leads to  flow equations that permit to decouple interacting quantum systems and provide equivalent  time-independent Hamiltonians for driven systems. With this approach, we can effectively demonstrate the equivalence between a time-periodic Hamiltonian and a quantum logic gate acting on a many qubit state. Since different periodic drives lead to different flow equations, the final logic gate will offer different outcomes on the same initial state. At certain resonances, we expect the flow equations to yield approximately vanishing transition amplitudes for all transitions in the renormalized Hamiltonian, leading to freezing. Away from these resonances, numerous phenomenologies are expected, especially in nonintegrable systems. These range from the onset of ergodicity via the eigenstate thermalization hypothesis (ETH), to Many Body Localization (MBL), a generalization of Anderson localization for interacting systems with disorder.

These analytical calculations will be supplemented by numerical methods, ranging from exact diagonalization and/or integration of the Heisenberg Dynamics, to approximate methods of varying degrees of accuracy. Exact dynamics is possible for large system sizes in integrable systems. Integrable systems that are disordered  can be block-diagonalized in SU(2N) (N is the system size) using Nambu spinors, and the Heisenberg equations of motion of the elementary creation operators can be formulated and solved numerically using high- performance distributed computing. The time-values of non-local responses can be obtained from a Pfaffian matrix of Wick - contracted expressions that represent quantum expectation values of correlation operators. For nonintegrable systems, such as Curie-Weiss magnets with long range interactions, as well as anisotropic transverse fields, my collaborators and I have been comparing various methods with exact dynamics for small system sizes, and are using them to study dynamics in larger sizes where exact dynamics is computationally prohibitive. Among the methods being benchmarked are the Discreet Truncated Wigner Approximation (dTWA), where the time dynamics of an observable of interest is expressed as a quantum-noise (described by the discreet Wigner function of the initial qubits) modulated dynamics of the corresponding classical observable (obtained from the Weyl symbol of the Hamiltonian), averaged over a discreet quantum phase space spanned by the qubits. The full average over the entire phase space is approximated by Monte-Carlo sampling. Discreet TWA is highly scalable computationally, making it more efficient than other approximate methods commonly used in nonequilibrium quantum dynamics, such as TEBD-DMRG and others. Our ongoing research indicates that dTWA can be enhanced by adding additional features, such as cluster expansions, as well as higher-order BBGKY corrections to the classical dynamics. We are extremely enthused about the possibility of applying this to periodically driven systems, and a major part of this project involves investigating the results obtained therein. The comparison of analytical and numerical results at the freezing condition will provide a robust and convincing description of entanglement freezing, and will be a significant step forward in our ability to correct quantum errors.

The final aspect of this project is the study of the time-evolution of an entangled many body state, such as the GHZ state,  as the system is time-periodically driven. The renormalized Hamiltonian, obtained by the techniques described above, is expected to provide us with characteristic time-scales that govern the onset of freezing, as well as its exact dynamical mechanism for entangled states. The physical quantities whose time-evolutions are of interest vis. a vis. quantum entanglement include non-local correlations, as well as Renyi and von-Neumann entanglement entropies, entanglement negativity, and numerous other entanglement measures and witnesses. While the onset of many body freezing is the primary objective of this project, an important secondary objective vis-a-vis the time evolution of a spatially delocalized entangled many body state involves the competition between many body freezing and the MBL state. When a disordered and strongly interacting system is driven out of resonance, the equivalent time-independent Hamiltonian, obtained by the renormalization group method described above, can undergo a so-called MBL transition at sufficiently low energy densities and strong disorder. In this regime,  the dynamics of a strongly entangled state can localize spatially, leading to dramatic changes in the information content that can be controlled in such a way as to induce the behavior of specific logic gates. Adiabatically tuning the disorder can lead to higher mobilities and correspondingly different logical outcomes on the input qubits. MBL is in stark contrast to thermalization, where the information content is lost completely. The interacting Aubre-Andre Hamiltonian is a good candidate for the study of the MBL phase once it is renormalized for periodic drives. An important objective of this project is to understand the nature of a periodically driven MBL state in the context of quantum information theory, and the dynamical study of both entanglement witnesses, contrasted with observables commonly used to probe the MBL transition, is of significant interest. As an example, bipartite entanglement , expressed by the Renyi Entropy, is a well-known proxy for whether a part of the system can act as a good heat bath for the rest. Thermalization would imply near-complete loss of quantum information due to errors, and a Renyi entropy that is characteristic of thermalized states. An MBL transition, however, would imply large but subthermal entropies that lead to growth in entanglement. Thus, the onset of quantum errors due to thermalization can be eliminated in a controlled way by inducing an MBL transition in the renormalized time independent Hamiltonian. Thus, adapting existing paradigms of the study of MBL to asymptotically renormalized Hamiltonians from periodic driving will prove highly useful towards the ultimate goal of fault tolerant quantum computing.

\section{Future Plans and Perspective:}

Future directions include the possibility that the time evolution of post-quenched qubit states may be ergodic in nature. A nonintegrable closed quantum system is believed to thermalize after a quench provided that the initial state is sufficiently delocalized in the Hilbert space of the quantum system. This thermal state is often described using the Eigenstate Thermalization Hypothesis, which postulates that the spectrum of time-averaged observables is sufficiently 'typical' i.e. delocalized in the Hilbert Space, that the probability distribution is smooth in energy, allowing for a microcanonical ensemble description. A hallmark of this transition are exponential decays in responses (obvious for open systems, but less so for closed ergodic ones). Nonintegrable systems can be modeled using quantum magnets similar to the ones described above, where the hopping is extended beyond nearest neighbours.Recent theoretical studies indicate that even closed integrable systems, such as the ones we plan to study with 
disordered nearest-neighbour hopping, may show such decays to the steady state. This may allow for providing a thermodynamic description of quantum errors in such systems. Finally, we plan to extend the study of error propagation to data transcription. This involves the copying of quantum data from one many body system to another. While this claim is somewhat controversial in academia, we plan to study the transcription of the ensuing errors by studying nonequilibrium dynamics in spin heterostructures, such as coupling Ising magnet systems with different  anisotropies.  A key system of interest is a transverse field Ising system coupled with an Emch-Radin system. Due to inherent symmetries of an Emch-Radin Hamiltonian, the quenched evolution of that subsystem is relatively simple. Modelling the coupling as a single tunneling bond would not affect the complexity much, and the techniques described above, refined over the course of this project, will help solve this problem.

The very important fields of quantum information and quantum nonequilibrium dynamics have generated new interest over the last decade, in large part, due to advances in experiments involving ultracold atoms. These highly tunable systems make it possible to study nonequilibrium dynamics in regimes hitherto inaccessible in solid state systems. The early experiments involving nonequilibrium ultracold atoms were carried out in the erstwhile Soviet Union and North America (spurred, in part, by cold war competition), followed by significant developments in Europe. The detailed study of heterostructures in a many body system approaching equilibrium, necessary in the study of quantum logic gates,  is a subset of the above-mentioned studies. Numerical approaches require computational facilities that were realized early on in the United States, with Europe following closely. It is only relatively recently that distributed computing environments in India have approached their counterparts in the West in performance metrics. So far, the bulk of the computational 
resources have been devoted to military, financial and biological research. As a result, only a few research groups in India, have forayed into the detailed study of nonequilibrium dynamics in established theoretical many body models.  Any advancements in the theoretical understanding of such dynamics will stimulate research in nonequilibrium dynamics in India, and improve the country's contributions to the study of quantum computing. Even a marginal success of the project \textit{viz.} a proof-of-concept of quantum error correction, would certainly lay the foundation stone of new paradigm in Indian science, where the physics of classical and quantum nonequilibrium systems would be used to revolutionize high performance computing.



\end{document}
