\documentclass[a4paper,9pt]{article}

%% A template for the IEF Marie Curie action
%
% All very simple code, and standard packages. The bibliography uses
% IEEEtranSA style, which is very similar to alpha.
%
% The ethical issues tables in section B6 could be forced in place
% more elegantly, but it worked for me.
%
% DOUBLE CHECK the details of your call before using this
% template. The name of the sections and subsections changes from call
% to call, and new sections are added and removed.
%
% August 2010, v1.0 - Jesus Nuevo-Chiquero.
%
% This file is provided AS IS, with absolutely no warranty of
% anything. You are welcome to use it, but you assume all risks.
\usepackage[defaultsans]{droidsans}
\renewcommand*\familydefault{\sfdefault} %% Only if the base font of the document is to be typewriter style
\usepackage[latin1]{inputenc}
\usepackage[margin=0.5in]{geometry}
\usepackage{setspace}
\usepackage[numbers, comma, sort&compress]{natbib}
\usepackage{hyperref}

\let\oldthebibliography=\thebibliography
  \let\endoldthebibliography=\endthebibliography
  \renewenvironment{thebibliography}[1]{%
    \begin{oldthebibliography}{#1}%
      \setlength{\parskip}{0ex}%
      \setlength{\itemsep}{0ex}%
  }%
  {%
    \end{oldthebibliography}%
  }

 \title{Research Proposal:\\ Evolution and Correction of Quantum Errors in Nonequilibrium Many Body Systems }
 \author{Analabha Roy\\CSIR Senior Research Associate,\\ Saha Institute of Nuclear Physics,\\ Kolkata, India}
 \date{\today}

\begin{document}
 \maketitle
\section{Non-Technical Summary}
The primary goal of this project is to understand and minimize via external control the occurrence of Quantum Error in Quantum Information Science and Computing. Quantum error correction is used in quantum computing to protect quantum information from errors due to disorder-dynamics, decoherence and other types of noise. Quantum computers make direct use of phenomena arising from quantum physics, such as linear superposition and quantum entanglement, to perform logical instructions on datasets. These differ from classical computers where instructions and data are encoded into bits that are always in one of two states. In quantum computers, a single quantum bit (qubit) can be a superposition of two states, and can be used to store significantly more information that its classical counterpart. Quantum computing is a fledgling subject, although experiments have been carried out on small qubit systems. Governments and military all over the world support research into this field. 

Governments and military all over the world support research into this field. Their eventual goal is to develop fault-tolerant quantum computation, enhancing the power of computers by many orders of magnitude leading to a unprecedented revolution in computers. Examples include quantum cryptography and cryptanalysis, inherent parallelism (allowing quantum computers to perform millions of computations simultaneously), powerful simulators for complex systems (maybe even the early universe itself), and even emergent artificial intelligence. All these things motivate our study of quantum error correction. A key quantity of interest in systems of many interacting quantum bits is one of Entanglement. Entangled states are quantum states superposed from the elementary constituents of a many body system in such a way that the quantum state of each component cannot be described independently of the others. These entangled many body states are believed to reduce quantum errors.  Thus, the generation and control of entangled qubit states is of enormous relevance to quantum computing. However, the onset of Nonequilibrium Dynamics in many qubit systems may cause information errors to propagate in time.

Nonequilibrium dynamics - the time-evolution of unstable states that disallow static thermodynamic descriptions of macroscopic physical systems, are of tremendous interest to condensed matter physicists and statistical mechanics people. A recently observed phenomenon in this area, that of Dynamical Many Body Freezing, causes prepared quantum states to freeze to their equilibrium values if the system is periodically driven at a known resonance. This occurs due to Quantum Interference, a linear superposition of quantum states that can cause responses to grow and fall in time, much like the bow waves from the wakes of many different boats moving in parallel criss-cross over water. A great many such interfering terms can cancel out all time evolution (the crest of one wave falling on the trough of another), and lead to freezing (calm waters). The main goal of this project is to investigate if freezing can be used to minimize the degradation of quantum information in many body systems. This is not obvious currently, since entangled qubit states may have unusual responses to driven dynamics that have not been studied so far. Other goals involve the study of the actual propagation of errors in time. One possibility is for the errors to manifest themselves as heat, allowing for a thermal description. Another is the study of Coarsening  viz. the slow growth of equilibrium structures as a time-independent system is prepared in qubit state and allowed to evolve in a regime with a distinct equilibrium phase. We also anticipate the utilization of our results in the understanding of quantum transcription, the potential ability to move quantum data from one system to another by coupling them. 

These studies will require the use of extensive analytical and numerical techniques which are described in the section below. My own interest in nonequilibrium dynamics stems from my undergraduate years, and I have been working with great passion in this field throughout my graduate and postdoctoral years. I have worked with many luminaries in this area, both in India as well as in the United States, and have developed an extensive network of academics who work in nonequilibrium dynamics. The opportunity to join RRI as a researcher will give me the ability to achieve the goals described above much more efficiently. I will present the results in national and international conferences in the BRIC countries, Europe, North America and Southeast Asia. I should also be able to host frequent academic visits of my colleagues to RRI, as well as get funding for academic visits of my own to them. I will deliver seminars that will increase interest in this area among undergraduate and graduate students, attracting potential research collaborators to RRI.  Given the importance that many academics, as well as people in the industry and government, attach to the prospects of quantum computing, this research project will help keep RRI at the final frontier of modern technology.

\section{Technical Description}
The main objective of this research project is to understand the onset and propagation of Quantum Errors in many particle systems from the perspective of nonequilibrium many body physics. Quantum errors can arise due to the degradation of the physical information in the state of a quantum system as it is evolved in time. A coherent quantum state can be used to store substantially more information than a system of classical bits of comparable size. Unlike discrete classical 0,1 digital states, a quantum bit (qubit) is continuous, describable by a superposition of states that can be characterized by a continuous parameter in the range [0,1], such as the direction on a Bloch sphere. The propagation of quantum error can be minimized by spreading the information in one qubit onto a highly-entangled state of several (physical) qubits.   However, a closed system of such qubits, evolving in time by the Schroedinger equation, will affect the stored coherent information. At  T=0, if the many body state is an 
eigenstate of the Hamiltonian of the system, then, in a time-independent Hamiltonian, only phase-evolution occurs . However, the advent of disorder and open links to external reservoirs means that the system is never in a true eigenstate, and therefore has components in the true eigenstates. Each component evolves in time with phase. The study of such dynamics is part of the field of quantum Nonequilibrium Dynamics - the description of quantum systems that can no longer be completely characterized by macroscopic time-independent quantities. When out of equilibrium in a time-independent Hamiltonian, the interference between the time-evolving eigenphases can cause decoherence and loss of information, manifesting themselves as a particular type of quantum error. Other kinds of quantum errors may arise due to quantum shot noise, as well as absorption by the environment in open quantum systems. Therefore, the study and minimization of such errors is of key interest.

The dynamics of quantum systems out of equilibrium has attracted a lot of theoretical and experimental attention in recent days. However, theoretical descriptions are hindered by the fact that little is understood about the dynamics of even the simplest of the quantum systems subjected to the simplest kinds of time-dependent drives. For example, the exact dynamics of a two-level system subjected to a sinusoidal drive is still an unsolved problem, inspiring innovative approximations leading to only a partial understanding. Within many such simple setups, quantum interference can manifest itself through surprising effects which are quite counter-intuitive and unexpected. One such phenomenon, that of Coherent Destruction of Tunneling , can be used to halt the evolution of a simple system by quantum dynamical interference. A many-body version of this phenomenon, known as Dynamical Many Body Freezing (also Dynamical Many Body Localization), has recently been observed both theoretically and experimentally. Here, 
the evolution of certain simple quantum systems can be strongly suppressed by arranging massive coherent cancellations of transition amplitudes through an appropriate periodic drive.  Naively speaking, such dynamical many-body freezing (as has been observed so far) is likely to be seen only in systems with simple underlying Hamiltonians. This is because the massive coherent cancellations of transition amplitudes that arise due to quantum interference seems possible only by fine-tuning a few parameters of a simple Hamiltonian. However, recent theoretical research suggests that a disordered many body system (which can be described theoretically by a very large number of random parameters) may also show freezing for sufficiently rapid drives while kept at a resonant condition. Note that this phenomenon is distinct from classical dynamical hysteresis, or Anderson localization in disordered systems. Thus, there is an exciting possibility that the evolution of quantum errors in a many body entangled state may be 
frozen at the desired value, eliminating this type of quantum error altogether!

The primary purpose of this project is to investigate the possibility that a multipartite entangled state may be completely 'frozen' in a manner described in the previous paragraph by applying a periodic drive at a known resonance condition. So far, theoretical research on freezing has focused on the evolution of driven systems of quantum magnets (where entangled states can be prepared), especially on Ising and Kitaev models with nearest-neighbour interactions in 1 and 2 dimensions respectively. Recent research has also drawn attention to this phenomenon in ultracold optical lattices, specifically the Bose-Hubbard model. However, only the time evolution of local responses, such as magnetization or particle density, have been studied and determined to be frozen at resonance. The evolution of non-local quantities, such as correlations or entanglement, is still an open problem. In this project, we plan to start from theoretical models that describe disordered quantum spin structures, such as the disordered 
Ising model (Edwards-Anderson model) in 1D, as well as generalizations to XX, XY and ZZ (Emch-Radin) models with transverse fields that break internal symmetries. The results can easily be carried over to the context of several other systems, e.g., those of hardcore bosons/free fermions in disordered potentials. 

The first objective is to study the time-evolution of an entangled many body state, such as the GHZ state,  as the system is time-periodically driven. Both analytical and numerical approaches will be used. Analytical approaches will involve adaptations of the Wilson Numerical Renormalization Group. Here, we will derive an  an effective time-independent Hamiltonian under a rotating wave approximation using a unitary flow-equation technique applied to periodically driven systems. This renormalized Hamiltonian is expected to provide us with characteristic time-scales that govern the onset of freezing, as well as its exact dynamical mechanism for entangled states. The physical quantities whose time-evolutions are of interest vis. a vis. quantum entanglement include non-local correlations, as well as Renyi and von-Neumann entanglement entropies, entanglement negativity, and numerous other entanglement measures. Our analytical studies will be complemented by numerical ones. Disordered spin structures can be block-
diagonalized in SU(2N) (N is the system size) using Nambu spinors, and the Heisenberg equations of motion of the elementary creation operators can be formulated and solved numerically using high- performance distributed computing. The time-values of non-local responses can be obtained from a Pfaffian matrix of Wick - contracted expressions that represent quantum expectation values of correlation operators. The comparison of analytical and numerical results will provide a robust and convincing description of entanglement freezing, and will be a significant step forward in our ability to correct quantum errors.

The next phase of our project will involve studying the actual propagation of quantum errors themselves, rather than trying to freeze them out. The time evolution of an entangled state in a closed time-independent Hamiltonian can be treated from the perspective of Coarsening. Coarsening describes the structural evolution of a many body system while it is approaching equilibrium after a quench in a  parameter; a particularly important aspect of the nonequilibrium dynamics seen in recent experiments. A 'quench' is defined as the diabatic variation of a thermodynamic or other  parameter of the system, such as temperature, mass or chemical potential. In a diabatic variation, the rate at which the quench rate is faster than all the relaxation rates intrinsic to the system, thus allowing for approximating the change as instantaneous. Coarsening after a quench constitutes the dynamical process by which the characteristic  size of the support of the equilibrium phases grows. During this coarsening process, space-
time correlations allow for the identification of a growing length scale. Domains of equilibrium states grow with this length scale, and a spatial profile of 'kinks' or  `domain walls' that demarcate these regions provides insights into the coarsening process. Thus, this part of the project proposes the investigation of such coarsening. The nonequilibrium dynamics involves principles like dephasing, quantum ergodicity, the Kibble-Zurek mechanism, Landau-Zener tunneling, nonequilibrium Schwinger-Keldysh theory, and others. This part of the research project will involve the application of some of the above-mentioned ideas, as well as newer stochastic methods, to Open Quantum Systems viz. systems that can exchange energy via connections to a reservoir of heat. For open systems, such dynamics can be studied by completely specifying the nature of the reservoir, and treat the coupled system + reservoir as a closed quantum system, or by treating the reservoir stochastically. We plan to profile quantum correlations 
and coarse-grained responses whose spatial extent describes the coarsening of quantum errors. These are expected to grow towards a steady state. The stochastic approach, using Lindblad dynamics, is computationally simpler. The density matrix for such a mixed system can be solved using Density Matrix Renormalization Group methods. The ensuing dynamics  can be simulated numerically with novel approaches to the time evolved block decimation using DMRG, including the implementation of these algorithms in distributed grid computing environments using established parallel-programming paradigms.

Future directions include the possibility that the time evolution of post-quenched qubit states may be ergodic in nature. A nonintegrable closed quantum system is believed to thermalize after a quench provided that the initial state is sufficiently delocalized in the Hilbert space of the quantum system. This thermal state is often described using the Eigenstate Thermalization Hypothesis, which postulates that the spectrum of time-averaged observables is sufficiently 'typical' i.e. delocalized in the Hilbert Space, that the probability distribution is smooth in energy, allowing for a microcanonical ensemble description. A hallmark of this transition are exponential decays in responses (obvious for open systems, but less so for closed ergodic ones). Nonintegrable systems can be modelled using quantum magnets similar to the ones described above, where the hopping is extended beyond nearest neighbours.Recent theoretical studies indicate that even closed integrable systems, such as the ones we plan to study with 
disordered nearest-neighbour hopping, may show such decays to the steady state. This may allow for providing a thermodynamic description of quantum errors in such systems. Finally, we plan to extend the study of error propagation to data transcription. This involves the copying of quantum data from one many body system to another. While this claim is somewhat controversial in academia, we plan to study the transcription of the ensuing errors by studying nonequilibrium dynamics in spin heterostructures, such as coupling Ising magnet systems with different  anisotropies.  A key system of interest is a transverse field Ising system coupled with an Emch-Radin system. Due to inherent symmetries of an Emch-Radin Hamiltonian, the quenched evolution of that subsystem is relatively simple. Modelling the coupling as a single tunneling bond would not affect the complexity much, and the techniques described above, refined over the course of this project, will help solve this problem.

\end{document}
