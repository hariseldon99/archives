\documentclass[a4paper,10pt]{report}
\usepackage[utf8]{inputenc}
\usepackage{url}
\usepackage{bera}
\usepackage{amsmath}
\usepackage{amssymb}
\usepackage{amsfonts}
\usepackage{hyperref}
\usepackage[margin=0.5in]{geometry}
\usepackage[square, comma, numbers, sort&compress]{natbib}
% Title Page
\title{Research report for the period $2012-2013$.\\
Submitted to the CSIR HRDG in partial fulfillment of the\\ 
criteria for renewal of the Senior Research Associateship.
}
\author{Analabha Roy,\\
CSIR Senior Research Associate,\\
Theoretical Condensed Matter Physics Division,\\
Saha Institute of Nuclear Physics,\\
$1/$AF Bidhannagar, Salt Lake, Kolkata $700064$.\\
Email:\url{daneel@utexas.edu}.\\
CSIR SRA No: $13$($8531$-A)/$2011$-Pool.
}

\begin{document}
\maketitle

\begin{abstract}
This report documents the research work that the author has been doing in his capacity as a CSIR Senior Research Associate at the Saha Institute of Nuclear Physics, Kolkata. The report covers the period beginning November $30^{\rm th}$, $2012$ and ending on October $1^{\rm st}$, $2013$. The research report is divided into four parts. 
The first part is the introduction. The second part summarizes the research work done in the nonequilibrium dynamics of Fermi gases. The third part summarizes ongoing works on the nonequilibrium dynamics of the disordered Ising model, the Eigenstate Thermalization hypothesis in many body systems, and coarsening dynamics in many body systems.
In addition to summaries, each section presents future plans and ongoing extensions to the current work, as well as the status of peer-review of the current research. The final part lists various formal academic activities conducted by the author during this period, such as attending conferences and summer schools, and delivering seminars and colloquia at various institutions.
\end{abstract}
\section{\sc Introduction}
\label{sec:intro}
This report summarizes the research work done by the author during the period beginning November $30^{\rm th}$, $2012$ and ending on October $1^{\rm st}$, $2013$ at the TCMP division of SINP, Kolkata under the CSIR Senior Research Associateship. The research work covers a wide range of subjects in condensed matter theory, with a focus on the dynamics of quantum gases such as driven Ising and Kitaev models, Fermi-BCS superfluids, and gauge Bosons. The research presented in this report is extremely relevant to condensed matter theorists and experimentalists seeking to emulate the dynamics of many particle systems using ultracold atoms in regimes inaccessible via solid state systems. The relevance also extends to those investigating coherent many particle dynamics and emergent behaviour of Bose Einstein Condensates and Fermi superfluids. Specific areas of relevance include problems involving quenches across the BCS-BEC crossover, quenching in the Bose Hubbard model, the study of near equilibrium properties of superconductors, the dynamical characteristics of pairing symmetries in BCS systems, the persistence of universal behaviour in nonequilibrium systems, and others. The ongoing research work presented below is also of interest to condensed matter physicists working on the problem of Eigenstate Thermalization, attempting to link classical models of thermalization in quenched gases to their quantum counterparts.

The organization of the report is as follows. Section $2$ presents research undertaken during the early months of the period covered in this report. This work is concerning the nonequilibrium dynamics of BCS systems, and continues the research conducted during previous period of the associateship. The focus is on periodically driven quenches modeled by Bogoliubov dynamics. The section also details the efforts directed towards a comprehensive review of research works done in this area, extending it to impulse quenches in BCS systems. The efforts culminated in a review article that the author was invited to write by the conveners of the National Conference on Nonlinear Systems and Dynamics that was held in IISER Pune in mid-$2012$, and has been published in the European Physical Journal, Special Topics. Section $3$ covers ongoing works on several topics that are being done in collaboration with many colleagues and the author as the primary contributor. The first project involves the dynamics of coarsening in classical and quantum systems, and proposes an exhaustive look at the transition to equilibrium of conserved classical $\phi^4$ theories, as well as systems of coupled macromagnets, via the growth of equilibrium structures. This project also proposes the study of the dynamics of open quantum systems using Keldysh methods, as well as the Lindblad formalism and Density Matrix Renormalization Group techniques. The next project involves the dynamics of the driven - disordered Ising model near exotic freezing, and the investigation of responses in that regime when translational symmetry is broken. The final project is looking at the dynamical transition towards quantum ergodicity in nonintegrable quantum many body systems as they are quenched away from equilibrium. 

\section{\sc  Nonequilibrium dynamics of ultracold Fermi superfluids}
\label{sec:bcs}
This project is a continuation of research works done under this associateship in the previous period. During the previous period, the author and his collaborators~\footnote{The collaborators are: Prof. Krishnendu Sengupta,  Dr. Arnab Das and Sanhita Modak of IACS, Kolkata, and Dr Raka Dasgupta of the Asia Pacific Center for Theoretical Physics, Pohang, Gyeongbuk, Korea.} studied the zero temperature non-equilibrium dynamics of a fermionic superfluid in
the BCS limit and in the presence of a drive leading to a time-periodic chemical potential $\mu(t)$. This was done in the collisionless regime, where the Bogoliubov excitations relax much more slowly than the order parameter, and the collision integrals appearing in the nonequilibrium Dyson equation of a driven BCS Hamiltonian can be ignored. Then, the Bogoliubov dynamics of the particle-hole amplitudes can be built from the Keldysh Greens function~\cite{gorkov:volkov}. This dynamics can be formulated as a series of coupled nonlinear Schr\"odinger equations, linked by the equilibrium BCS gap equation continued in time. The research demonstrated that the BCS self-consistency condition is crucial in shaping the long-time behaviour of the fermions subjected to the drive and provided an analytical understanding of the behaviour of the fermion density $n_{{\mathbf k}_F}$ (where ${\mathbf k}_F$ is the Fermi momentum vector) after a drive period and for large $\omega$. In addition, the researchers investigated the effects of the self-consistency on certain universal behavior seen in non self-consistent many particle dynamics, such as those of the Ising and Kitaev models. The researchers observed the disappearance of exotic freezing of the degrees of freedom that has been seen for driven Ising models~\cite{arnab1}. They also noted the persistence of the Kibble-Zurek scaling in the self-consistent dynamics, for d-wave pairing symmetry. The Kibble-Zurek mechanism leads to a universal scaling law that arises near quantum critical points due to the formation of topological defects in many body systems as they are dynamically driven~\cite{zurek,bikashbabu}. This mechanism produces a defect density profile that the researchers suggest can be used to experimentally detect pairing symmetries in high $T_c$ superconductivity. The work was subjected to an extensive peer-review process and significant additions were done over the course of $2013$, resulting in publication earlier this year in the Journal of Condensed Matter Physics~\cite{mypaper2}

In addition, the author was invited to write a review article by the conveners of the National Conference on Nonlinear Systems and Dynamics (NCNSD)~\cite{ncnsd}. The conference was held in IISER Pune in $2012$~\cite{ncnsd}, where the author presented his works detailed above. The review article highlighted the researcher's contributions in the area of nonequilibrium BCS dynamics, and went beyond  to survey the literature on the study of nonequilibrium dynamics of Fermi  superfluids in the BCS and BEC limits, both in the single channel and dual channel cases. The focus remained on mean field approaches to the dynamics, with specific attention drawn to the dynamics of the Ginzburg-Landau order parameters of the Fermi and composite Bose fields, as well as on the microscopic dynamics of the quantum degrees of freedom. The two  approaches are valid approximations in two different time scales of the ensuing dynamics. The system is presumed to evolve during and/or after a quantum quench in the parameter space. The quench can either be an impulse quench with virtually instantaneous variation, or a periodic variation between two values. The literature for the order parameter dynamics, described by the time-dependent Ginzburg- Landau equations, is reviewed, and the works of the author in this area highlighted. The mixed phase regime in the dual channel case is also considered, and the dual order parameter dynamics of Fermi-Bose mixtures reviewed. Finally, the nonequilibrium dynamics of the microscopic degrees of freedom for the superfluid is reviewed for the self-consistent and non self-consistent cases. The dynamics of the former can be described by the Bogoliubov de-Gennes equations with the equilibrium BCS gap equation continued in time and self-consistently coupled to the BdG dynamics. The latter is a reduced BCS problem and can be mapped onto the dynamics of Ising and Kitaev models. The author reviewed the dynamics of both impulse quenches in the Feshbach detuning, as well as periodic quenches in the chemical potential. The work involved extensive literature search and reading, as well as setting up and exploring techniques, presented as suggestions for future endeavors in this area. The review article was evaluated and accepted for publication in the NCNSD conference proceedings, presented in a special edition of the European Physical Journal~\cite{myreview}.

\section{\sc Ongoing works}
\label{sec:ongoing}

\subsection{Coarsening in classical and quantum systems}
\label{subsec:ccqs}
This project entails a long term collaboration in an international research team consisting of the author and several other academics.  The lead researcher is Prof Leticia F. Cugliandolo of the Laboratoire de Physique Th\'eorique et Hautes Energies, Universit\'e Pierre et Marie Curie - Paris VI, France. Other members include Profs. G. Lozano and L. Arrachea (Universidad de Buenos Aires, Argentina), D. Rossini  and R. Fazio (Scuola Normale Superiore, Pisa, Italy). The proposal for this research project developed during the $2013$ US- India Advanced Studies Institute on Thermalization, a workshop held in IISc Bangalore which was attended by the author and the lead researcher. The main objective of this research project is to understand the dynamics of {Coarsening} in {Classical} and {Quantum} many body {Systems} while they are approaching equilibrium after a quench in a  {parameter}; a particularly important aspect of the nonequilibrium dynamics seen in recent experiments. The focus lies on classical nonequilibrium dynamical systems in a closed field theory with a double well potential, the classical evolution of an open system like coupled micromagnets, and the nonequilibrium dynamics of quantum-quenched dissipative spin heterostructures. Coarsening after a quench constitutes the dynamical process by which the characteristic  {size} of the support of the equilibrium phases grows after the diabatic quench in a system parameter.

The first part of the proposed project involves the analytical and numerical study of classical quenches in $\phi^4$ theory in  {$d$} 
dimension. The potential in the free energy will be constructed to $4^{th}$ order in the order parameter $\phi$. The evolution of the free-energy landscape with the control parameter driving a phase transition guides the understanding of the post-quench dynamics from, typically, a disordered phase to an ordered phase. The project involves the study of quenches to criticality, as well as sub-critical quenches, by investigating the microscopic dynamics of the order parameter. This dynamics is governed by a quasi-Newtonian equation of motion with,  {for example, a thermalized ensemble of initial states in a free-energy landscape with a single minimum at zero field}. Such systems can be solved numerically by conventional quadrature and finite element methods, or more modern symplectic integrators appropriate for Hamiltonian systems~\cite{symplectic}.  Other descriptions, such as the field large-$\mathcal{N}$ approximation, will also be considered.  The first question to be explored is whether the system re-thermalizes to a steady state or not. This can be tested via the fluctuation-dissipation theorem, where linear responses to a small external field are compared to the field correlations. Subsequently, the temporal behavior of the equilibrating field will be studied. During the \textit{coarsening} process, space-time correlations allow for the identification of a growing length scale. Domains of equilibrium states are expected to grow with this length scale, and a spatial profile of 'kinks' or  {`domain walls'} that demarcate these regions is expected to provide insights into the coarsening process. Domain walls will be identified, whose gradients are expected to drive coarsening. Coarsening is thus expected to drive the nucleation and growth of domains that support the equilibrium phase. Near criticality, divergences of time scales via critical slowing down is also known to occur, and the scaling behavior of critical quenches will also be studied analytically using scaling and renormalization group arguments. In addition, the system is expected to build fractal clusters~\cite{fractal} and the equilibrium and nonequilibrium contributions are multiplicatively separated. In sub-critical quenches, the asymptotic behavior of the characteristic coarsening scale relative to equilibrium correlations is expected to be governed by the dynamic scaling hypothesis~\cite{dynscal}, where the domain structure is statistically independent of time when lengths are scaled accordingly. The goals of this phase of the project are the profiling of the universal scaling laws described above, as well as the evolution of structure interfaces in the order parameter field through the kinks in the solution.

The next phase of the project will involve studying the dynamics of open classical systems \textit{viz.} systems connected to a thermal reservoir. Classical $\phi^4$ theories of the type discussed above can be linked to closed Ising magnets, and open systems of magnets can be studied by dealing with the interactions between magnetic moments on sub-micrometre length scales. These are governed by competition between the dipolar energy and the exchange energy, the outcome of which governs the long range magnetic order, if any. In such systems, the Landau-Lifshitz-Gilbert (LLG) equation is a key model for describing the nonlinear evolution~\cite{gll:review}. The dynamics of many coupled LLG systems connected to a thermal bath (canonical ensemble at equilibrium) can be 
formulated from here. In the continuum limit, the result is a nonlinear partial differential equation in the spin field. The ensuing dynamics links the physics to the microscopic Hamiltonian structures of spin lattices, including planar XX and XY structures~\cite{gll:review} whose quantum evolution will be studied later. This phase of the project aims to formulate the LLG dynamics after a quench across the magnetic phase transitions, and investigate the transition to equilibration in a manner similar to that described in the paragraph above. The dynamics in these systems have wide interdisciplinary relevance. They have intimate connections with many of the well-known integrable soliton equations, including nonlinear Schr\"odinger and sine-Gordon equations~\cite{gll:review,sinegordon}. The possibility of classical chaos in such systems~\cite{gll:review} leads it to other disciplines, such as power generation and synchronization~\cite{lax13} and inhomogeneous filaments~\cite{lax14}. 

The logical continuation of this project now draws attention to the quantum realm in such systems. The quantum dynamics part of this research project will involve the application of some of methods used in closed quantum dynamics described in section~\ref{sec:bcs}, as well as newer stochastic methods, to \textit{open quantum systems} \textit{viz.} systems that can exchange energy via connections to a reservoir of heat. Away from $T=0$, the pure quantum dynamics has to be weighed by the thermal probabilities of the initial state that is presumed to be in thermal equilibrium. For open systems, such dynamics can be studied by completely specifying the nature of the reservoir, and treat the coupled system + reservoir as a closed 
quantum system, or by treating the reservoir stochastically. The nonequilibrium dynamics, thus formulated, allows the profiling of local quantum correlations and coarse-grained responses whose 
spatial extent described coarsening that is expected to grow towards the steady state. Analytics will involve modeling such systems by \textit{spin chain heterostructures}~\cite{arrachea},  constructed by linking finite or semi-infinite XX or XY spin chains (of various anisotropies) to each other at their ends. The dissipative dynamics of the central chain after a quench can thus be obtained from the Hamiltonian quantum dynamics of the entire system of links. Such systems can be mapped onto p-wave BCS superconducting fermions, and the nonequilibrium Dyson equation can be formulated in a manner similar to the nonequilibrium BCS problem in section~\ref{sec:bcs}~\cite{gorkov:volkov}. Post-quench dynamics can be studied by solving the resultant equations for the local fermion correlations by diagrammatic approximations to the nonequilibrium self-energy, either in the collisionless regime as in section~\ref{sec:bcs}, or in the regime where collisions are rapid enough to have relaxed away and only order parameter dynamics remains~\cite{rammer, colrev,myreview}.

The approach detailed above, with the full Hamiltonian description of the system and reservoir, comes at the price of numerous analytical and computational difficulties. The final part of this project will deal with alternative formalisms of open quantum systems, where the reduced density matrix is evolved in time by a master equation, allowing for the inclusion of incoherent processes which represent interactions with a reservoir. Such dynamics is regularly studied in quantum optics using the \textbf{Kossakowski-Lindblad equation}, where it can represent absorption or emission from a reservoir~\cite{lindblad}. Here, the regular Liouvillean dynamics of closed quantum density matrices are dressed with Lindblad bath operators which act locally on degrees of freedom near each bath. Although this approach is not universally valid, it is a reasonable starting point to study Heisenberg chains such as the ones being studied in this project~\cite{lindblad}. Here, as in the previous paragraph, the ends of 
a finite spin chain are coupled to canonical Lindblad spin operators whose amplitudes are determined by the thermodynamics of the reservoir.  This approach is computationally simpler, and the density matrix for such a mixed system can be solved using DMRG methods. The dynamics of coarsening in Lindblad systems can be solved after a quench in this manner.

Coarsening is a very basic aspect of non-equilibrium quantum dynamics. Understanding it would shine light on a vast landscape of quantum phenomena ranging from the process of defect generation in critical quantum relaxation~\cite{relaxation}, thermalization of a closed many-body quantum system~\cite{ rigol:nature:etc}, thermalization and effective temperatures of open quantum systems~\cite{thermopen}, glassy systems~\cite{glassy}, mesoscopic systems~\cite{meso}, to the operation of near future quantum devices like an analog quantum computer~\cite{arnab1}. The potential and the target of our project, as well as its methodologies, are thus genuinely interdisciplinary and of very broad interest.

\subsection{Universal behavior in the driven disordered Ising model}
This project involves a collaboration with Profs Krishnendu Sengupta and Arnab Das of the Theoretical Physics Department, IACS, Kolkata, as well as Kasturi Basu, a graduate student at the same institute. The primary objective is to study the nonequilibrium quantum dynamics of a disordered 1d Ising model as it is driven back and forth across its quantum critical point. We aim to provide a physical description of the time-disorder averaged state and expectation values of responses. At zero disorder, the system is expected to show certain universal dynamical features at particular parameter ranges. These include adiabatic dynamics for small frequencies, Landau Zener tunneling for drives with frequencies that are close to the energy gap, and exotic freezing for larger drive frequencies~\cite{arnab1}. The onset of disorder at the freezing points is expected to affect this behavior, and induce thermalization to a steady state glassy phase for sufficiently strong disorder. The immediate goal of this project is to formulate a numerical algorithm for the integration of an Ising chain in a disordered transverse field, as well as a disordered exchange coupling. The transverse field has a controlled time-periodic component of known amplitude and frequency. The numerical algorithm integrates the dynamics of the Hamiltonian,
\begin{equation} \label{H_OBC}
H(t) = - \sum_i^{L-1} J\left(1+\alpha J_i\right) \{c^{\dagger}_i c^{\dagger}_{i+1} + c^{\dagger}_i c_{i+1}  + {\rm H.c.}\} 
    - 2 \sum_i^{L} \left[\Gamma(t)+\alpha h_i\right] c^{\dagger}_i c_i \;,
\end{equation}
whose equivalence to a 1D driven disordered Ising model can be demonstrated by a Jordan-Wigner Transformation~\cite{isingrand}.
Here, $\Gamma(t)$ can be $\Gamma_0\cos{\omega t}$ or $\Gamma_0 \left(t/\tau\right)$ as in~\cite{isingrand}. The quantities $J_i$ and $h_i$ are random numbers, either uniform in the range $(-\sigma,\sigma)$, or Gaussian with a standard deviation of $\sigma$. The fluctuations $\sigma$ are scaled to unity and time is measured  in units of $\hbar/\sigma$. The Hamiltonian can be written as a sum of the time-driven Ising Hamiltonian $ H_D(t)$ and a disordered Ising Hamiltonian whose contribution is controlled by a weak perturbative term $\alpha$. Thus,
\begin{eqnarray}
H(t)&=& H_D(t)+\alpha H_R,\nonumber \\
H_D(t) &=& -J\sum_i^{L} \left(c^{\dagger}_i c^{\dagger}_{i+1} + c^{\dagger}_i c_{i+1}  + {\rm H.c.}\right) - 2 \Gamma(t)\sum_i^{L}c^{\dagger}_i c_i,\nonumber \\
H_R &=& -\sum_i^{L} J J_i \left(c^{\dagger}_i c^{\dagger}_{i+1} + c^{\dagger}_i c_{i+1}  + {\rm H.c.}\right) - 2 \sum_i^{L}h_i c^{\dagger}_i c_i.
\end{eqnarray}
The objective is to do a general study of the model dynamics, and investigate the dynamics of an arbitrary initial state when $H_D(t)$ is at resonance and exotically freezes the unperturbed Hamiltonian~\cite{arnab1}. The full model can be mapped to the following dynamics via Bogoliubov transformations~\cite{isingrand}.
\begin{eqnarray} \label{BdG_tdep:eqn}
i\frac{d}{dt}u_{i\mu}(t) \!\! &=&  
{2} \sum_{j=1}^{L} \left[A_{i,j}(t)u_{j\mu}(t)+B^o_{i,j}(t)v_{j\mu}(t) \right] 
\nonumber \\
i\frac{d}{dt}v_{i\mu}(t) \!\! &=& \!\! 
-{2}\sum_{j=1}^{L} \left[A_{i,j}(t)v_{j\mu}(t)+B^o_{i,j}(t)u_{j\mu}(t) \right] 
\;,
\end{eqnarray}
Here $A$ and $B^o$ are real $L\times L$ matrices. The matrix $A=A^d(t)+A^o$, where $A^d_{i,j}(t)=-\left[\Gamma(t)+\alpha h_i\right]\delta_{i,j}$ is diagonal. The
non-zero elements of $A^o$ and $B^o$ are given by $A^o_{i,i+1}=A^o_{i+1,i}=-\left[J\left(1+\alpha J_i\right)\right]/2$, $B^o_{i,i+1}=-B^o_{i+1,i}=-\left[J\left(1+\alpha J_i\right)\right]/2$. Furthermore, the system state is characterized by two \textit{complex} $L\times L$ matrices $u$ and $v$, whose columns are the $L$-dimensional vectors $u_\mu$ and $v_\mu$ for $\mu=1,\dots,L$. The algorithm built by the researchers numerically integrates the dynamics described above using a parallel multi-threaded API. It also evaluates the magnetization and magnetization time correlations as functions of time, and dumps the output to file. 

The magnetization per site can be shown via Jordan Wigner transformation~\cite{arnab1} to be
\begin{equation}
\label{eq:magdef}
m(t)\equiv 1 - \frac{2}{L}\sum^L_{i=1} \langle \psi(0) | c^\dagger_i (t) c_i(t) |\psi(0)\rangle,
\end{equation}
where the operators are in the Heisenberg picture, and $|\psi(0)\rangle$ is the initial state at $t=0$. Note that the number operator is NOT conserved by the Hamiltonian in eq~\ref{H_OBC}. At $t=0$, this Hamiltonian can be diagonalized by the following Bogoliubov operators,
\begin{eqnarray}
\label{eq:initdiag}
\gamma_\mu &\equiv& \sum^L_{j=1} \left(u^\ast_{j\mu}c_j+v^\ast_{j\mu}c^\dagger_j\right), \nonumber \\
c_i &=& \sum^L_{\mu=1} \left(u_{i\mu}\gamma_\mu+v^\ast_{i\mu}\gamma^\dagger_\mu\right).
\end{eqnarray}
The dynamics in eqns~\ref{BdG_tdep:eqn} continues this transformation in time in the Heisenberg picture, thus
\begin{equation}
\label{eq:cit}
c_i(t) = \sum^L_{\mu=1} \left[u_{i\mu}(t)\gamma_\mu+v^\ast_{i\mu}(t)\gamma^\dagger_\mu\right].
\end{equation}
Plugging the equation above into eq~\ref{eq:magdef} yields
\begin{eqnarray}
\label{eq:magops}
 m(t) &=& 1-2 \langle\psi(0)|\hat{d}(t)|\psi(0)\rangle,\nonumber \\
 \hat{d}(t) &=& \frac{1}{L}\sum^L_{i,\mu,\nu = 1}\bigg[u^\ast_{i\nu}(t) u_{i\mu}(t)\gamma^\dagger_\nu\gamma_\mu + u^\ast_{i\nu}(t)v^\ast_{i\mu}(t)\gamma^\dagger_\nu\gamma^\dagger_\mu + \nonumber \\
  & & v_{i\nu}(t)u_{i\mu}(t)\gamma_\nu\gamma_\mu + v_{i\nu}(t)v^\ast_{i\mu}(t)\gamma_\nu\gamma^\dagger_\mu\bigg] .
\end{eqnarray}
Now, note that, in the last of eqs~\ref{eq:magops}, all but the final expression on the RHS are normal ordered. In general, the expectation value of $\hat{d}(t)$ wrt the initial state cannot be calculated without knowing the exact eigenspectrum of $H(0)$. However, if the initial state is an eigenstate of $H(0)$, specifically the ground state, then the expectation of all normal ordered terms in eqs~\ref{eq:magops} vanish. This leads to the simplification
\begin{equation}
 \label{eq:mag}
m(t) = 1 - \frac{2}{L} \times {\rm Tr}\left[{v}^\dagger(t) {v}(t)\right]. 
\end{equation}
Equation~\ref{eq:mag} will be used to profile the response of the system as it evolves in time from the initial state, which will be set to the ground state of $H(0)$. Note that different sets of random numbers will give different Hamiltonians initially, and any meaningful statistics must come from profiling over multiple disorder configurations.
Finally, the correlations are computed from the definition
\begin{equation}
\label{eq:corrdef}
C(t) \equiv \frac{1}{L}\sum^L_{i=1} \langle \psi(0) | c^\dagger_i (t) c_i(t) c^\dagger_i (0) c_i(0) |\psi(0)\rangle,
\end{equation}
Applying eqns~\ref{eq:cit} and~\ref{eq:initdiag} to the above, and assuming that $ |\psi(0)\rangle$ is the ground state of $H(0)$, 
yields the formula
\begin{equation}
\label{eq:corr}
C(t) = \frac{1}{L}\sum^L_{i\mu\alpha=1} \bigg[v_{i\nu}(t)u_{i\alpha}(t)u^\ast_{i\alpha}(0)v^\ast_{i\nu}(0)-
  v_{i\nu}(t)u_{i\alpha}(t)u^\ast_{i\nu}(0)v^\ast_{i\alpha}(0)+|v_{i\nu}(t)|^2|v_{i\alpha}(0)|^2\bigg].
\end{equation}
As of the date of submission of this report, the researchers have successfully compiled, tested and debugged the algorithm as described above, and are currently running it for parameters at the exotic freezing regime with different disorders to determine the nature of the transition from frozenness to a thermalized glass. The work is expected to yield concrete results by the end of the calendar year. 

\subsection{Eigenstate Thermalization Hypothesis and the transition to Quantum Ergodicity}
This project is an ongoing collaboration with Prof Arti Garg of the Saha Institute of Nuclear Physics, Kolkata. In this work, the researchers seek to contribute to recent discussions on the general relation between integrability and thermalization in the nonequilibrium dynamics of quantum many body systems. The most basic nonequilibrium process that is experimentally viable in quantum gases of ultracold atoms is the \textit{quantum quench}, following which the state of the system is allowed to evolve dynamically according to the Schr\"odinger equation. Recent studies have shown that that the long-time asymptotic behavior of a quantum system out of equilibrium is dependent on its integrability. Nonintegrable systems have been seen to \textit{thermalize}, where time averages of dynamical observables do not depend on the initial conditions~\cite{rigol:nature:etc}, but only on the initial energy. Such 'thermal' behavior cannot, in general, be seen in integrable systems, where additional initial state information is required in order to predict the dynamical steady state.

The Eigenstate Thermalization Hypothesis (or ETH) is an attempt to determine whether an isolated nonequilibrium
quantum many body system can be described using equilibrium statistical mechanics~\cite{rigol:nature:etc}. In classical many body systems, the dynamical transition to statistical equilibrium occurs due to chaos in the phase space, and is thus linked to classical nonintegrability by the KAM theorem. A chaotic system will spend equal time in equal areas of its phase space, sampling the entire phase space as it evolves in time. This mechanism provides an explanation for the success of equilibrium statistical mechanics. Assuming that the time over which laboratory experiments are performed is generally much longer than the time required for the system to sample a large portion of its phase space, then thermal averages are effectively time averages of the system's ergodic motion throughout phase space. However, quantum dynamical systems show unitary evolution via the Schr\"odinger equation, and thus can never be chaotic. Thus, the mechanism by which an isolated quantum many body system approaches a statistical description remains an open question. If a quantum system is populated at an initial state 
\begin{equation}
|\Psi(0)\rangle = \sum_\alpha C_\alpha |E_\alpha\rangle,
\end{equation}
where $|E_\alpha\rangle$ are the energy eigenstates of the Hamiltonian $H$, then the system evolves after an initial quench via the time independent Schr\"odinger equation, whose solution is
\begin{equation}
 |\Psi(t)\rangle = \exp{\left[-iHt\right]}|\Psi(0)\rangle.
\end{equation}
The expectation value of any observable $A$ is
\begin{equation}
\langle\Psi(t) |A|\Psi(t)\rangle = \sum_{\alpha\beta}C^\ast_\alpha C_\beta e^{-i\left(E_\beta-E_\alpha\right)t} \langle E_\alpha|A|E_\beta\rangle.
\end{equation}
The long time average of this expectation value simplifies to 
\begin{equation}
\bar{A} = \sum_\alpha |C_\alpha|^2 \langle E_\alpha|A|E_\alpha\rangle,
\end{equation}
once all the harmonic oscillatory contributions average out. This value is the predicted average in the \textit{diagonal ensemble}. The ETH states that, in thermalized systems, this diagonal ensemble value approaches the average described by the corresponding equilibrium microcanonical ensemble,  given by the equally-weighted average over all energy eigenstates within some energy window centered around the mean energy of the system~\cite{rigol:nature:etc} \textit{viz.}
\begin{equation}
\bar{A} \approx \langle A\rangle_{mc}\equiv \frac{1}{\mathcal N}\sum^{\mathcal N}_{\gamma=1} \langle E_\gamma|A|E_\gamma\rangle,
\end{equation}
where $\mathcal{N}$ is the number of states within the energy window. A critical aspect of this claim renders the verification of ETH to be a highly nontrivial problem. The diagonal average $\bar{A}$ depends on the choice of the initial state (through the $C_\alpha$s), whereas the microcanonical average $\langle A\rangle_{mc}$ makes absolutely no reference to the initial state of the system. It is expected~\cite{rigol:nature:etc} that the equivalence only happen for certain classes of Hamiltonians and initial states, such as the Hamiltonian have fewer symmetries than degrees of freedom, and the initial state attain typicality \textit{i.e} be highly delocalized in the Hilbert space spanned by the eigenstates~\cite{rigol:nature:etc}. Currently, there is no known derivation of the Eigenstate Thermalization Hypothesis for general systems. However, it has been verified to be true for a wide variety of interacting systems using numerical exact diagonalization techniques, to within the uncertainty of these methods. 

The researchers have successfully replicated the numerical simulation of a tightly bound 2D lattice of hard sphere bosons as described in~\cite{rigol:nature:etc} and verified the ETH hypothesis when the system is prepared in a spatially localized state and allowed to diffuse into the lattice. The current goal of the project is to investigate the initial state dependence on the dynamical transition to this thermalized state. In classical thermal systems, sensitive dependence on initial conditions cause dynamical chaos which can be profiled by the Lyapunov exponent. This measure characterizes the rate of separation of infinitesimally close trajectories. A positive Lyapunov exponent is usually taken as an indication that the system is chaotic, and a many body system with a positive Lyapunov exponent is thus expected to thermalize. The researchers are currently attempting to provide a similar description for closed quantum many body problems. This entails a well-defined measure of separation between two Hilbert space states that is infinitesimal at $t=0$ and grows with time as the nonintegrable quantum system evolves after an impulse quench. Despite the unitary and non chaotic nature of quantum dynamics, the high degree of delocalization in Hilbert space for the 'typical' states that are expected to thermalize may be sufficient to induce a sensitive dependence of initial conditions. Currently, numerical simulations of such dynamics for the 2D hard sphere problem are under way, and concrete results are expected by the end of the calendar year. 

\section{\sc Seminars, Conferences and Workshops:}
This section lists all the formal academic activities of the author that were conducted during the period covered in this report. The activities constitute delivering seminars at various academic institutions in India and internationally, attending an international summer workshop on nonequilibrium many body physics, and academic visits to colleagues in the researcher's field of expertise. As a result of these activities, the visibility of the researcher's works have increased, common interests with other scientists in this area have been discussed and the foundations for future collaborations have been established.
 
\begin{itemize}
\item 
{\bf May 2013} Invited seminar: Department of Physics, Hong Kong University. Research group of Prof S. Zhang
\item 
{\bf July 2013} US-India Advanced Studies Institute on Thermalization: From Glasses to Black Holes: {Indian Institute of Science, Bangalore}\\
Summer workshop on thermalization: conceptual foundations to modern-day applications in complex condensed matter systems, quantum information theory, and string theory.
\item
{\bf August 2013}: Invited seminar: Indian Institute of Science Education and Research, Pune, India. Research group of Prof. G. Ambika.
\item
{\bf August 2013}: Invited seminar: Bhabha Atomic Research Centre, Mumbai, India. Research group of Prof. S. R. Jain. 
\end{itemize}

\begin{thebibliography}{10}

\bibitem{gorkov:volkov}
L.P. Gor’kov, G.M. Eliashberg, Sov. Phys. JETP \textbf{27}, 328 (1968);
A.F. Volkov, Sh.M. Kogan, Sov. Phys. JETP \textbf{38}(5), 1018 (1974). Available online at \url{http://jetp.ac.ru/cgi-bin/e/index/e/38/5/p1018?a=list}.

\bibitem{arnab1} A. Das, Phys. Rev. B, {\bf 82}, 172402 (2010); S. Bhattacharyya, A. Das and S. Dasgupta, PRB {\bf 86}, 054410 (2012); Arnab Das and B. K. Chakrabarti Eds., \textit{Quantum Annealing and Related Optimization Methods}, Lecture Note in Physics, {\bf 679}, Springer-Verlag, Heidelberg (2005).

\bibitem{zurek}
\newblock W.H. Zurek, 
\newblock\textit{Cosmological experiments in condensed matter systems}. 
\newblock Phys. Rep. {\bf 276}(4): 177 (1996).
\newblock DOI:\url{http://dx.doi.org/10.1016/S0370-1573(96)00009-9}.

\bibitem{bikashbabu}
A. Dutta, U. Divakaran, D. Sen, B.K. Chakrabarti, T.F. Rosenbaum, G. Aeppli, arXiv:1012:0653, (2010).

\bibitem{mypaper2}
\newblock Analabha Roy, Raka Dasgupta, Sanhita Modak, Arnab Das, and Krishnendu Sengupta, 
\newblock  J. Phys: Condens. Matter {\bf 25}, 205703 (2013).
\newblock arXiv: \url{http://arxiv.org/abs/1209.4144}.

\bibitem{ncnsd}
The National Conference on Nonlinear Systems and Dynamics (NCNSD) website can be found at \url{http://www.ncnsd.org}. Details on the $2012$ conference are at \url{http://www.ncnsd.org/next.php}.

\bibitem{myreview}
\newblock A. Roy, 
\newblock Invited mini-review, 
\newblock Eur. Phys. J. ST, {\bf 222} (3-4), 975-993 (2013). 
\newblock arXiv:\url{http://arxiv.org/abs/1211.6936}

\bibitem{symplectic}
E. Forest and R.D. Ruth, Physica D {\bf 43} 105 (1990); H. Yoshida, Phys. Lett. A {\bf 150} (5-7):262 (1990).

\bibitem{fractal}
J.D. Gunton, M. San Miguel, and P.S. Sahni, in \textit{Phase Transitions and Critical Phenomena}, C. Domb
and J.L. Lebowitz eds. (Academic Press, New York, 1983)  {\bf 8} 267; H. Furukawa, Adv. Phys. {\bf 6}, 703 (1985);
J. Langer, in \textit{Solids Far From Equilibrium}, C. Godr'eche ed. (Cambridge University Press, Cambridge, 1992).

\bibitem{dynscal}
B.I. Halperin and P.C. Hohenberg, Phys. Rev. {\bf 177}:2, 952 (1969).

\bibitem{gll:review}
M. Lakshmanan,  Phil. Trans. R. Soc. A {\bf 369}:1939 1280-1300 (2011); M. Lakshmanan and A. Saxena, Physica D {\bf 237} 885-897 (2008); J. A. G. Roberts and C.J. Thompson, J. Phys. A {\bf 21} 1769-1780 (1988).

\bibitem{sinegordon}
M. Daniel and L. Kavitha, Phys. Rev. B {\bf 66} 184433 (2002); H.J. Mikeska and M. Steiner, Adv. Phys. {\bf40}:3 191-356 (1991). 

\bibitem{lax13}
J. Grollier, V. Cross  and A. Fert, Phys. Rev. B {\bf 73} 060409 (2006).

\bibitem{lax14}
Y.B. Bazaliy, B.A. Jones and S.C. Zhang, Phys. Rev. B {\bf 69} 094421 (2004).

\bibitem{arrachea}
L. Arrachea, G. S. Lozano, and A. A. Aligia, Phys. Rev. B {\bf 80}, 014425 (2009).

\bibitem{rammer}
J. Rammer, \textit{Quantum Field Theory of Non-equilibrium States} (Cambridge University Press, Cambridge 2007).

\bibitem{colrev}
A. Roy, Eur. Phys. J. {Plus}, {\bf 127}:3, 34 (2012).

\bibitem{lindblad}
A. Kossakowski, Rep. Math. Phys. {\bf 3} 247 (1972); G. Lindblad , Commun. Math. Phys. {\bf 48} 119 (1976);S. Clark, J. Prior, M. J. Hartmann, D. Jaksch, and M. B. Plenio, New J. Phys. {\bf 12}, 025005 (2010).

\bibitem{relaxation}
W. H. Zurek, U. Dorner, and P. Zoller, Phys. Rev. Lett. {\bf 95}, 105701 (2005).

\bibitem{rigol:nature:etc}
\newblock M. Rigol, V. Dunjko,and M. Olshanii.
\newblock Nature {\bf 452}, 854 (2008);
\newblock A. Polkovnikov, K. Sengupta, A. Silva, M. Vengalattore.
\newblock Rev Mod Phys {\bf 83}, 863 (2011);
\newblock M. Srednicki, 
\newblock Physical Review E \textbf{50} (2), 888 (1994).

\bibitem{thermopen}
A. Caso, L. Arrachea, G. S. Lozano, Eur. Phys, J B, {\bf 85}:266, (2012).

\bibitem{glassy}
L. F. Cugliandolo and J. Kurchan, Phys. Rev. Lett. {\bf 71}, 173-176 (1993).

\bibitem{meso}
L. Arrachea and L. F. Cugliandolo, Europhys. Lett. 70 642 (2005).

\bibitem{isingrand}
T. Caneva, R. Fazio, and G.E. Santoro, Phys. Rev. B {\bf 76}, 144427 (2007).

\end{thebibliography}

\end{document}          
